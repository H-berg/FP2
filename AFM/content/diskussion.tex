\section{Diskussion}
\label{sec:Diskussion}
%aufgabe1
Die im ersten Aufgabenteil vermessenen Strukturgr\"o{\ss}en der verschiedenen Mikrostrukturen stimmen gut mit den im Datenblatt angegebenen Werten \"uberein.
In Tabelle (\ref{tab:auf1_disk}) sind die Messergebnisse inklusive Abweichung nochmals aufgef\"uhrt.

\begin{table}
	\centering
	\caption{Messergebnisse der Mikrostrukturen im Vergleich zu den Kenngr\"o{\ss}en aus dem Datenblatt \cite{sample}.}
\begin{tabular}{|r|ccc|}
	\hline
	{} & {Kreis} & {Quadrat} & {Streifen} \\
	\hline
	Struckturabstand / $\mu m$ & $4,865 \pm 0,109$ & $ 9,797 \pm 0,149$ & $4,869 \pm 0,072$ \\
	Datenblatt / $\mu m$ & 5 & 10 & 5 \\
	Abweichung / \%	& 2,7 & 2,03 & 2,62 \\
	\hline
\end{tabular}
\label{tab:auf1_disk}
\end{table}

%aufgabe2
\bigskip
Die Vermessung der drei verschiedenen Speichermedien in der zweiten Versuchsreihe wurde mit dem selben Auswertungsprogramm, welches schon f\"ur den ersten Versuchsteil verwendet wurde, durchgef\"uhrt.
Die Fehler der Mittelwerte aus Tabelle (\ref{tab:auf2}) lassen auch hier auf gute Messwerte schlie{\ss}en.
Da alle Abst\"ande und Gr\"o{\ss}en der Pits auf den Speichermedien manuell mit dem Distance-Tool vermessen wurden, ist es schwer zu sagen, wie genau diese Werte sind.
Denn beispielsweise die AFM-Aufnahmen der Blu-ray (siehe Anhang) sind im Verglich mit den aufgenommenen Bilder der CD unscharf.
So lassen sich die R\"ander der Pits auf der Blu-ray nicht exakt bestimmen.
Zus\"atzlich wurde in diesem Aufgabenteil die Speicherkapazit\"at der CD abgesch\"atzt.
Wie in Tabelle (\ref{tab:auf2_disk}) gezeigt ist, liegt diese berechnete Speicherkapazit\"at \"uber der gew\"ohnlich Speicherkapazit\"at einer CD.
Die Abweichung liegt zum einen an den Abweichungen der gemessenen Pitl\"angen und -abst\"anden.
Beispielsweise wurde f\"ur die Bestimmung der minimalen Pitl\"ange ausschlie{\ss}lich der Mittelwert aus sechs Messungen berechnet.
%Auch die Vorgehensweise zur Absch\"atzung der Speicherkapazit\"at birgt ungenauigkeiten, die zu Abweichungen f\"uhren k\"onnen.
%Es wurde angenommen, dass die kleinste Strecke zwischen zwei H\"ohen\"uberg\"angen einer Folge von vier Bit entspricht.
Au{\ss}erdem handelt es sich hier lediglich um eine Absch\"atzung und nicht um eine genau Berechnung.
\begin{table}
	\centering
	\caption{Vergleich der Abgesch\"atzten Speicherkapazit\"at der untersuchten CD.}
\begin{tabular}{|ccc|}
	\hline
	{Kapazit\"at exp.} & {Kapazit\"at theo.} & {Abweichung} \\
	\hline
	$1081,17 \,$MB & $900 \,$MB & $20,13 \,$\% \\
	\hline
\end{tabular}
\label{tab:auf2_disk}
\end{table}

%aufgabe3
\bigskip
Die gemessenen Kraft-Abstandskurven weisen alle einen typischen Verauf auf.
Der Snap-Out Bereich ist jedoch nicht bei allen untersuchten Materialien deutlich erkennbar.
So konnte bei Teflon dieser Werte nicht exakt aus der Grafik (siehe Abbildung (\ref{abb:teflon})) entnommen werden.
Die berechneten Adh\"asionskr\"afte liegen alle in der gleichen Gr\"o{\ss}en{\-}ord{\-}nung.
Zur Bestimmung des Elastizit\"atsmoduls der Teflon-Probe wurde die Cantileverspitze als kegelförmig angenommen.
Der Öffnungswinkel und die Federkonstante des Cantilevers sowie die Poissonzahl wurden aus Literaturangaben gewonnen.
Im Vergleich mit dem theoreitsch erwarteten Wert f\"ur das Elastizit\"atsmodul ist die experimentelle Berechnung sehr gut, siehe Tabelle(\ref{tab:auf3d}).
\begin{table}
	\centering
	\caption{Vergleich des Elastizit\"atsmoduls der Teflon-Probe mit dem Literaturwerte \cite{teflon}.}
\begin{tabular}{|ccc|}
	\hline
	{Elastizit\"atsmodul exp.} & {Elastizit\"atsmodul theo.} & {Abweichung} \\
	\hline
	$474,83 \,$MB & $420 \,$MB & $13,05 \,$\% \\
	\hline
\end{tabular}
\label{tab:auf3d}
\end{table}
Die Abweichung des Elastizit\"atsmodul ist durch den nicht linearen Verlauf des Kontaktbereichs der Edelstahlprobe zu erkl\"aren.
Diese wurde bei der Bestimmung der Eindringsteife als als Referenzprobe verwendet.
Die Eindringtiefe $d$ in die Teflon-Probe wurde hier deim maximale Verfahrweg $z_p$ der Probe gemessen.

\bigskip
Insgesamt ist das Rasterkraftmikrokop (AFM) eine gute Variante zur Bestimmung von Topographien einer makroskopisch glatten Oberfläche sowie deren Materialeigenschaften.
