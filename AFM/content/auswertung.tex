\section{Auswertung}
\label{sec:Auswertung}

\subsection{Messung der Topografie einer Mikrostruktur}
Im Verlauf der ersten Versuchsreihe werden verschiedene Strukturformen auf einer Mikrostrukturprobe betrachtet.
Mittels des AFM-Aufbaus werden von den drei vorhandenen Strukturen AFM-Bilder aufgenommen.
Es wird jeweils eine $(20 \times 20) \, \mu m$ gro{\ss}e Fl\"ache vermessen.
In den Abbildungen (\ref{abb:kreis}) bis (\ref{abb:streif}) sind diese Bilder zu sehen.

\begin{figure}[H]
\centering
	\begin{subfigure}[t]{0.45\textwidth}
	\includegraphics[width=\textwidth]{AFM_auswertung/Kreis_durch_vor.png}
	\caption{Vermessung des Kreisdurchmessers.}
	\label{abb:kreisa}
	\end{subfigure}
	~
	\begin{subfigure}[t]{0.45\textwidth}
	\includegraphics[width=\textwidth]{AFM_auswertung/Kreis_abs_vor.png}
	\caption{Bestimmung des Abstands zwischen den Kreisen.}
	\label{abb:kreisb}
	\end{subfigure}
\caption{Vermessung der Kreisstruktur auf der Mikrostrukturprobe. Aufgenommen wurde eine Fl\"ache der Gr\"o{\ss}e von $(20 \times 20) \, \mu m$. Zur Vermessung der jeweiligen Strecken wurde das 'distance and directinos-Tool' des Auswertungsprogramm Gwyddion verwendet.}
\label{abb:kreis}
\end{figure}


\begin{figure}[H]
\centering
	\begin{subfigure}[t]{0.45\textwidth}
	\includegraphics[width=\textwidth]{AFM_auswertung/quad_durch_vor.png}
	\caption{Vermessung der Quadratgr\"o{\ss}e.}
	\label{abb:quada}
	\end{subfigure}
	~
	\begin{subfigure}[t]{0.45\textwidth}
	\includegraphics[width=\textwidth]{AFM_auswertung/quad_abb_vor.png}
	\caption{Bestimmung des Abstands zwischen den Quadraten.}
	\label{abb:quadb}
	\end{subfigure}
\caption{Vermessung der Quadratstruktur auf der Mikrostrukturprobe. Aufgenommen wurde eine Fl\"ache der Gr\"o{\ss}e von $(20 \times 20) \, \mu m$. Zur Vermessung der jeweiligen Strecken wurde das 'distance and directinos-Tool' des Auswertungsprogramm Gwyddion verwendet.}
\label{abb:quad}
\end{figure}


\begin{figure}[H]
\centering
	\begin{subfigure}[t]{0.45\textwidth}
	\includegraphics[width=\textwidth]{AFM_auswertung/streif_durch_vor.png}
	\caption{Vermessung der Streifenbreite.}
	\label{abb:streifa}
	\end{subfigure}
	~
	\begin{subfigure}[t]{0.45\textwidth}
	\includegraphics[width=\textwidth]{AFM_auswertung/streif_abb_vor.png}
	\caption{Vermessung des Abstands zwischen zwei Streifen.}
	\label{abb:streifb}
	\end{subfigure}
\caption{Vermessung der Streifen auf der Mikrostrukturprobe. Aufgenommen wurde eine Fl\"ache der Gr\"o{\ss}e von $(20 \times 20) \, \mu m$. Zur Vermessung der jeweiligen Strecken wurde das 'distance and directinos-Tool' des Auswertungsprogramm Gwyddion verwendet.}
\label{abb:streif}
\end{figure}


\begin{table}
	\centering
	\caption{Messwerte der Mikrostrukturprobe. Aufgelistet sind die gemittelten Werte der vermessenen Strecken und Abst\"ande. Der Strukturabstand ergibt sich aus der Summer von Gr\"o{\ss}e und Abstand der Mikrostruktur.}
\begin{tabular}{|r|ccc|}
	\hline
	{} & {Kreis} & {Quadrat} & {Streifen} \\
	\hline
	Größe / $\mu m$ & $3,177 \pm 0,096$ & $5,958 \pm 0,120$ & $2,037 \pm 0,039$ \\
	Abstand / $\mu m$ & $1,688 \pm 0,052$ & $3,839 \pm 0,089$ & $2,832 \pm 0,061$ \\
	Struckturabstand / $\mu m$ & $4,865 \pm 0,109$ & $ 9,797 \pm 0,149$ & $4,869 \pm 0,072$ \\
	Datenblatt / $\mu m$ & 5 & 10 & 5 \\
	Abweichung / \%	& 2,7 & 2,03 & 2,62 \\
	\hline
\end{tabular}
\label{tab:auf1}
\end{table}


\subsection{Topographie einer CD, DVD und Blu-ray}

\begin{figure}[H]
\centering
	\begin{subfigure}[t]{0.45\textwidth}
	\includegraphics[width=\textwidth]{AFM_auswertung/cd_3D.png}
	\caption{.}
	\label{abb:cd_3d}
	\end{subfigure}
	~
	\begin{subfigure}[t]{0.45\textwidth}
	\includegraphics[width=\textwidth]{AFM_auswertung/dvd_3d.png}
	\caption{.}
	\label{abb:dvd_3d}
	\end{subfigure}
	\\
	\begin{subfigure}[t]{0.45\textwidth}
	\includegraphics[width=\textwidth]{AFM_auswertung/bluray_3d.png}
	\caption{.}
	\label{abb:br_3d}
	\end{subfigure}
\caption{.}
\label{abb:3d}
\end{figure}

\begin{figure}[H]
\centering
	\begin{subfigure}[t]{0.4\textwidth}
	\includegraphics[width=\textwidth]{AFM_auswertung/cd_breite.png}
	\caption{.}
	\label{abb:}
	\end{subfigure}
	~
	\begin{subfigure}[t]{0.4\textwidth}
	\includegraphics[width=\textwidth]{AFM_auswertung/cd_abstand.png}
	\caption{.}
	\label{abb:}
	\end{subfigure}
	\\
	\begin{subfigure}[t]{0.4\textwidth}
	\includegraphics[width=\textwidth]{AFM_auswertung/cd_Lmin.png}
	\caption{.}
	\label{abb:}
	\end{subfigure}
	~
	\begin{subfigure}[t]{0.4\textwidth}
	\includegraphics[width=\textwidth]{AFM_auswertung/cd_Lmax.png}
	\caption{.}
	\label{abb:}
	\end{subfigure}
\caption{.}
\label{abb:CD}
\end{figure}

\begin{figure}[H]
\centering
	\begin{subfigure}[t]{0.3\textwidth}
	\includegraphics[width=\textwidth]{AFM_auswertung/cd_tiefe.png}
	\caption{.}
	\label{abb:}
	\end{subfigure}
	~
	\begin{subfigure}[t]{0.3\textwidth}
	\includegraphics[width=\textwidth]{AFM_auswertung/dvd_tiefe.png}
	\caption{.}
	\label{abb:}
	\end{subfigure}
	~
	\begin{subfigure}[t]{0.3\textwidth}
	\includegraphics[width=\textwidth]{AFM_auswertung/bluray_tiefe.png}
	\caption{.}
	\label{abb:}
	\end{subfigure}
	\\
	\begin{subfigure}[t]{0.3\textwidth}
	\includegraphics[width=\textwidth]{AFM_auswertung/cd_tiefe_grafik.png}
	\caption{.}
	\label{abb:}
	\end{subfigure}
	~
	\begin{subfigure}[t]{0.3\textwidth}
	\includegraphics[width=\textwidth]{AFM_auswertung/dvd_tiefe_grafik.png}
	\caption{.}
	\label{abb:}
	\end{subfigure}
	~
	\begin{subfigure}[t]{0.3\textwidth}
	\includegraphics[width=\textwidth]{AFM_auswertung/bluray_tiefe_grafik.png}
	\caption{.}
	\label{abb:}
	\end{subfigure}
\caption{.}
\label{abb:pit_tiefe}
\end{figure}


\begin{table}
	\centering
	\caption{.}
\begin{tabular}{|r|ccc|}
	\hline
	{} & {CD} & {DVD} & {Blueray} \\
	\hline
	Abstand / $\mu m$ & $1,459 \pm 0,031$ & $0,778 \pm 0,014$ & $0,314 \pm 0,008$ \\
	Breite	/ $\mu m$ &	$0,454 \pm 0,023$ & $0,153 \pm 0,009$ & $0,102 \pm 0,008$ \\
	Minimale Länge / $\mu m$ & $0,656 \pm 0,025$ & $0,299 \pm 0,022$ & $0,098 \pm 0,007$ \\
	Maximale Länge / $\mu m$ & 2,127 & $0,858 \pm 0,022$ & 0,436 \\
	Pittiefe /  &  &  &  \\
	\hline
\end{tabular}
\label{tab:auf2}
\end{table}


\subsection{Federkonstante, Adhäsionskraft, Elastizitätsmodul mittels Kraft-Abstandskurven}

\begin{figure}[H]
\centering
	\begin{subfigure}[t]{0.3\textwidth}
	\includegraphics[width=\textwidth]{AFM_auswertung/edelstahl_kurve.png}
	\caption{.}
	\label{abb:}
	\end{subfigure}
	\\
	\begin{subfigure}[t]{0.3\textwidth}
	\includegraphics[width=\textwidth]{AFM_auswertung/teflon_kurve.png}
	\caption{.}
	\label{abb:}
	\end{subfigure}
	\\
	\begin{subfigure}[t]{0.3\textwidth}
	\includegraphics[width=\textwidth]{AFM_auswertung/TiN_kurve.png}
	\caption{.}
	\label{abb:}
	\end{subfigure}
\caption{.}
\label{abb:auf3}
\end{figure}

