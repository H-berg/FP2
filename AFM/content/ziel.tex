\section{Ziel}
\label{sec:Ziel}

Das Rasterkraftmikroskop wurde 1986 von Gerd Binning, Calvin Quate und Christoph
Gerber \cite{entw} erfunden und erlaubt eine räumliche Auflösung in der Größenordnung von
wenigen zehntel Pikometer. Die Auflösung ist damit mindestens genauso gut, wie die
des von Christoph Gerber mitentwickelten Rastertunnelmikroskops mit dem großem
Vorteil, nun auch nichtleitende Proben vermessen zu können. Zusätzlich zeichnet
sich das Rasterkraftmikroskop durch einen weiteren Vorteil aus: Der Handlichkeit.
Denn zur Vermessung einer Probe ist kein Vakuum nötig, wodurch das Rasterkraftmikroskop
bereits auf einen handelsüblichen Tisch verwendet werden kann. Anwendung findet das
Rasterkraftmikroskop deshalb in sämtlichen Naturwissenschaften sowie in der
Industrie.\\
In diesem Versuch wird die Topografie einer Siliziumdioxid-Probe mit unterschiedlichen
Mikrostrukturen, sowie verschiedene Speichermedien: CD, DVD und Blu-ray mittels
des Rasterkraftmikroskops im Kontakt-Modus untersucht. Anhand einer
Kraft-Abstandskurve von Edelstahl, Teflon und einer TiN-Schicht wird abschließend
das jeweilige Elastizitätsmodul und die Adhäsionskräfte zwischen Messspitze und
Probe bestimmt.
