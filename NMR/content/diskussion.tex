\section{Diskussion}
\label{sec:Diskussion}
Zunächst werden auf die im Unterkapitel \ref{sec:Abstimmung} bis \ref{sec:T2}
bestimmten Einstellparameter eingegangen.
Bei der Abstimmung des Probekopfes wird etwa $1,4\,\%$ der Eingangsleistung
$P_{\text{T}}$ reflektiert, womit das Magnetfeld der RF-Spule genügen stark sein
sollte. Ein besserer Probekopf würde denoch vermutlich ein VSWR $\approx1$ erzielen
können.\\
Im Unterkapitel \ref{sec:phase} wurde die Phase $\varphi$ bestimmt. Dabei fällt
in Abbildung \ref{fig:phase} auf, dass die Signal-Amplitude im Minimum (etwa bei $5\,°$)
betragsmäßig größer ist als das Maximum (etwa bei $185\,°$), woraus sicht deuten
lässt, dass das Mess-Signal um $180\,°$ gedreht ist.
Auch bei der Bestimmung der Pulslänge $t_{\pi}$ im Unterkapitel \ref{sec:pulslaenge}
zeigt sich in Abbildung \ref{fig:pulslaenge}, dass das Mess-Signal gedreht ist.
Denn das Mess-Signal verhält sich wie eine Sinus-Funktion, welche um $180\,°$
verschoben ist.\\
Die im Unterkapitel \ref{sec:T1} bestimmte $T_1$-Zeit liegt bei
$T_1 = \SI{18.313\pm0.350}{\milli\second}$ und für den Exponenten $b$ der
Kohlrauschfunktion ergibt sich ein Wert von $b = \SI{0,546\pm0,004}{}$. $b$ liegt
dem optimalen Wert von eins nah. Bei $b$=1 würde sich die longitudinale
Magnetisierung $M_{\text{z}}$, wie in der Theorie unter Gleichung \ref{eq:Mz}
beschrieben, verhalten. In Abbildung \ref{fig:T1} zeigt sich das typische Verhalten
der $T_1$-Relaxation. Nachdem die Magnetisierung $M$ in die transversale Ebene gekippt ist,
kann anfangs kaum longitudinale Magnetisierung $M_{\text{z}}$ gemessen werden.
Erst nachdem einige Millisekunden vergangen sind, kann bedingt durch die Spin-Gitter
Relaxation eine Magnetisierung $M_{\text{z}}$ gemessen werden. Nach der Evolutionszeit
$t=T_1$ hat sich etwa $63\,\%$ der Magnetisierung aufgebaut. Schließlich ist nach
ungefähr $\SI{100}{\milli\second}$ die longitudinale Magnetisierung $M_{\text{z}}$
etwa auf den ursprünglichen Wert relaxiert ist.\\
Die $T_2$ wird im Unterkapitel \ref{sec:T2} bestimmt und ergibt eine Spin-Spin
Relaxationszeit von $\SI{0.160\pm0.011}{\milli\second}$.Damit sei die Bedingung:
$T_1 \gg T_2$ erfüllt.
Für den Exponenten ergibt sich ein Wert von $b = \SI{-1,359\pm0,173}{}$, welcher
stark von der Theorie abweicht. Hier ist nach Gleichung ?? ein Exponenten $b$ von
eins zu erwarten gewesen. In Abbildung \ref{fig:T2} zeigt sich das Verhalten der
Spin-Spin Relaxation. Sobald die Magnetisierung $M$ in die transversale Ebene gekippt
ist, zeigt sich betragsmäßig das Signal am stärksten. Nachdem der Evolutionszeit
$t=T_2$ hat sich etwa $37\,\%$ der transversalen Magnetisierung $M_{\text{xy}}$
wieder aufgebaut.Aufgrund eines Problems mit dem Phasenzyklus wird hier der
Betrag des Mess-Signals: $|$Re$+$Im$|$ dargestellt. Damit ist das Problem leider
nur teilweise behoben, denn in Abbildung \ref{fig:T2} bei einer Evolutionszeit
von ungefähr $\SI{0,6}{\milli\second}$ zeigt sich eine erneuerte Ansteigung des
Mess-Singals. Die ist vermutich auch der Grund für den fälschlichen Expontenten $b$.
Im weiteren ist es nicht bei jeder $T_2$-Messung möglich, eine Kohlrauschfunktion
an die Messwerte gelegt werden.\\
Zunächst werden die beiden Abbildungen \ref{fig:sin} und \ref{fig:cos} aus den
stimulierten Echo Messungen verglichen. Der Kurvenverlauf der beiden Messungen
unterscheiden sich. Die $sin$-$sin$-Messung in Abbildung \ref{fig:sin}
zeigt sofort am Anfang einen starken Abfall. Hingegen zeigt die $cos$-$cos$-Messung
anfangs einen sehr schwachen Abfall. Bedingt durch die Dynamik bei er Umoriertierung
des Dimethylsulfon Moleküls wäre ein schneller Zerfall am Anfang zu erwarten.
Dieser zeigt sich hingegen nur bei der $sin$-$sin$-Messung. Im weiteren Kurvenverlauf
zeigt sich bedingt durch die eintretende Dekorrelation eine Änderung der Steigung.
Bei der $cos$-$cos$-Messung etwa nach $\SI{3}{\milli\second}$ und bei der
$sin$-$sin$-Messung nach etwa $\SI{0.5}{\milli\second}$, wobei sich bei der
$sin$-$sin$-Messung daraufhin ein leichtes Plateau bildet.
Zuletzt zeigen beide Kurven wieder ein abfallendes Verhalten, wobei die Steigung
des $cos$-$cos$-Abfalls relativ steil ist und die $sin$-$sin$-Messung einen eher
milderen Abfall zeigt. Aufgrund der Spin-Gitter Relaxation wäre hier ein eher
schwacher Zerfall, wie $sin$-$sin$-Messung zu erwarten gewesen. Im Fall, dass die
Mischzeit $t_{\text{m}} \rightarrow\infty$, streben beide Funktionen gegen den
Fitparameter $S_0$, welche eine Ähnliche Amplitude besitzen ($S_{0_{\text{cos}}} = \SI{-0.067\pm0.003}{}$ und
$S_{0_{\text{sin}}} = \SI{-0.080\pm0.01}{}$).\\
Es fällt besonders auf, dass sich die beiden Korrelationszeiten $\tau$ (siehe
Tabelle \ref{tab:tau}) stark unterscheiden. Der Bedingung: $T_2 \ll \tau \ll T_1$
zufolge, ist die Korrelationszeit $\tau_{\text{cos}}$ der sinnvollere Wert.
Alledings haben die Messung der $T_2$-Zeit Probleme bereitet, weswegen man
nicht gewissenhaft auf den Wert vertrauen kann. Hingegen zeigt sich bei der
$sin$-$sin$-Messung ein Kurvenverlauf, der dem der Versuchanleitung \cite{Anleitung}[S.25]
ähnlich ist.\\
Die beiden Spin-Gitter-Relaxationszeiten sind ebenfalls im Tabelle \ref{tab:tau}
aufgeführt. Die als Fitparameter freigegebene $T_{1\text{,Q}}$-Zeit ist dabei mehr
als doppelt so lang, wie die $T_1$-Zeit.

\begin{table}
  \centering
  \caption{Korrelationszeit $\tau$ aus dem Unterkapitel \ref{sec:stecho}}
  \label{tab:tau}
  \sisetup{table-format=1.2}
  \begin{tabular}{S S S S}
    \toprule
    {$\tau_{\text{cos}}$ / ms} & {$T_1$ / ms} & {$\tau_{\text{sin}}$ / ms} & {$T_{1\text{,Q}}$ / ms}\\
    \midrule
    {1,810\pm0,084} & {18,313\pm0,350} & {0,253\pm0,019} & {42,754\pm3,911}\\
    \bottomrule
  \end{tabular}
\end{table}
\noindent
\\
Bei der Untersuchung der Temperaturabhängigkeit im Unterkapitel \ref{sec:tempabh}
konnte in Abbildung \ref{fig:tempabh} für die $cos$-$cos$- und $sin$-$sin$-Messungen
ein linear Anstieg der Korrelationszeit $\tau$ mit kleiner werdenden Temperatur
beobachtet werden. Denn bei kleineren Temperaturen ist bedingt durch den
Boltzmann-Faktor $\exp{(-E/K_{\text{B}}T)}$, die Sprungrate kleiner als bei
höheren Temperaturen. Es zeigt sich in den $cos$-$cos$- und $sin$-$sin$-Messungen
eine annähert übereinstimmende Aktivierungsenergie $E$.
Hingegen liegt der Unterschied zwischen den beiden Vorfaktoren $\tau_{0,\text{cos}}$
und $\tau_{0,\text{sin}}$ bei einer Größenordnung (siehe Tabelle \ref{tab:tempabh}).
Grund dafür mag das oben diskutierte unterschiedliche Verhalten zwischen den
$cos$-$cos$- und $sin$-$sin$-Messungen sein.\\
Die Spin-Gitter Relaxationszeit in Abbildung \ref{fig:tempabh} wird hingegen bei
kleineren Temperaturen kürzer, da auch diese dem Arrehenius-Gesetz folgt. Die
Spin-Spin-Relaxationszeit konnten leider nicht genauer untersucht werden (s.o.).

\begin{table}
  \centering
  \caption{Aktivierungsenergie $E$ und Vorfaktor $\tau_0$ aus Unterkapitel
  \ref{sec:tempabh}}
  \label{tab:tempabh}
  \sisetup{table-format=1.2}
  \begin{tabular}{S S S S}
    \toprule
    \multicolumn{2}{c}{cos-cos} & \multicolumn{2}{c}{sin-sin} \\
    {E / eV} & {$\tau_0$ / $\SI{1e-13}{\second}$} & {E / eV} & {$\tau_0$ / $\SI{1e-13}{\second}$} \\
    \midrule
    {0,912\pm0,051} & {27,258\pm1,476} & {0,831\pm0,012} & {515,100 ± 7,239} \\
    \bottomrule
  \end{tabular}
\end{table}
\noindent
\\
In dem Spektrum in Abbildung \ref{fig:spek} ist deutlich ein Dublet zu erkennen,
welches durch die beiden Deuteronen verursacht werden. Der Abstand zwischen den
Extremstellen beschreibt dabei ihre Kopplungskonstante, welche zum einen bei
$\SI{39,51}{\kilo\hertz}$ und zum anderen bei $\SI{44,66}{\kilo\hertz}$ liegt.
Im weiteren Auslauf der beiden benannten Extremstellen, sind kleine Extremstellen
zu beobachten. Da Kohlenstoff $^{12}C$ einen Kernspin $I$ von Null hat, rührt
dieser Spektralanteil von dem Sauerstoff $O$ und dem Schwefel $S$.
