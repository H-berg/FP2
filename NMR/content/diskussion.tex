\section{Diskussion}
\label{sec:Diskussion}
Zunächst werden auf die im Unterkapitel \ref{sec:Abstimmung} bis \ref{sec:T2}
bestimmten Einstellparameter eingegangen.
Bei der Abstimmung des Probekopfes wird etwa $1,4\,\%$ der Eingangsleistung
$P_{\text{T}}$ reflektiert, womit das Magnetfeld der RF-Spule genügen stark sein
sollte. Ein besserer Probekopf würde denoch vermutlich ein VSWR $\approx1$ erzielen
können.\\
Im Unterkapitel \ref{sec:phase} wurde die Phase $\varphi$ bestimmt. Dabei fällt
in Abbildung \ref{fig:phase} auf, dass die Signal-Amplitude im Minimum (etwa bei $5\,°$)
betragsmäßig größer ist als das Maximum (etwa bei $185\,°$), woraus sicht deuten
lässt, dass das Mess-Signal um $180\,°$ gedreht ist.
Auch bei der Bestimmung der Pulslänge $t_{\pi}$ im Unterkapitel \ref{sec:pulslaenge}
zeigt sich in Abbildung \ref{fig:pulslaenge}, dass das Mess-Signal gedreht ist.
Denn das Mess-Signal verhält sich wie eine Sinus-Funktion, welche um $180\,°$
verschoben ist.\\
Die im Unterkapitel \ref{sec:T1} bestimmte $T_1$-Zeit liegt bei
$T_1 = \SI{18.313\pm0.350}{\milli\second}$ und für den Exponenten $b$ der
Kohlrauschfunktion ergibt sich ein Wert von $b = \SI{0,546\pm0,004}{}$. $b$ liegt
dem optimalen Wert von eins nah. Bei $b$=1 würde sich die longitudinale
Magnetisierung $M_{\text{z}}$, wie in der Theorie unter Gleichung \ref{eq:Mz}
beschrieben, verhalten. In Abbildung \ref{fig:T1} zeigt sich das typische Verhalten
der $T_1$-Relaxation. Nachdem die Magnetisierung $M$ in die transversale Ebene gekippt ist,
kann anfangs kaum longitudinale Magnetisierung $M_{\text{z}}$ gemessen werden.
Erst nachdem einige Millisekunden vergangen sind, kann bedingt durch die Spin-Gitter
Relaxation eine Magnetisierung $M_{\text{z}}$ gemessen werden. Nach der Evolutionszeit
$t=T_1$ hat sich etwa $63\,\%$ der Magnetisierung aufgebaut. Schließlich ist nach
ungefähr $\SI{100}{\milli\second}$ die longitudinale Magnetisierung $M_{\text{z}}$
etwa auf den ursprünglichen Wert relaxiert ist.\\
Die $T_2$ wird im Unterkapitel \ref{sec:T2} bestimmt und ergibt eine Spin-Spin
Relaxationszeit von $\SI{0.160\pm0.011}{\milli\second}$.Damit sei die Bedingung:
$T_1 >> T_2$ erfüllt.
Für den Exponenten $b$ ergibt sich ein Wert von $\SI{-1,359\pm0,173}{}$, welcher
stark von der Theorie abweicht. Hier ist nach Gleichung ?? ein Exponenten $b$ von
eins zu erwarten gewesen. In Abbildung \ref{fig:T2} zeigt sich das Verhalten der
Spin-Spin Relaxation. Sobald die Magnetisierung $M$ in die transversale Ebene gekippt
ist, zeigt sich betragsmäßig das Signal am stärksten. Nachdem der Evolutionszeit
$t=T_2$ hat sich etwa $37\,\%$ der transversalen Magnetisierung $M_{\text{xy}}$
wieder aufgebaut.Aufgrund eines Problems mit dem Phasenzyklus wird hier der
Betrag des Mess-Signals: $|$Re$+$Im$|$ dargestellt. Damit ist das Problem leider
nur teilweise behoben, denn in Abbildung \ref{fig:T2} bei einer Evolutionszeit
von ungefähr $\SI{0,6}{\milli\second}$ zeigt sich eine erneuerte Ansteigung des
Mess-Singals. Die ist vermutich auch der Grund für den fälschlichen Expontenten $b$.
Im weiteren ist es nicht bei jeder $T_2$-Messung möglich, eine Kohlrauschfunktion
an die Messwerte gelegt werden.\\
