\section{Diskussion}
\label{sec:Diskussion}
Zunächst wird auf die im Unterkapitel \ref{sec:Abstimmung} bis \ref{sec:T2}
bestimmten Einstellparameter eingegangen.
Bei der Abstimmung des Probekopfes wird etwa $1,4\,\%$ der Eingangsleistung
$P_{\text{T}}$ reflektiert, womit das Magnetfeld der RF-Spule genügend stark sein
sollte.\\
Im Unterkapitel \ref{sec:phase} wurde die Phase $\varphi$ bestimmt. Dabei fällt
in Abbildung \ref{fig:phase} auf, dass die Signal-Amplitude im Minimum (etwa bei $5\,°$)
betragsmäßig größer ist als das Maximum (etwa bei $185\,°$), woraus sicht deuten
lässt, dass das Mess-Signal um $180\,°$ gedreht ist.
Auch bei der Bestimmung der Pulslänge $t_{\pi}$ im Unterkapitel \ref{sec:pulslaenge}
zeigt sich in Abbildung \ref{fig:pulslaenge}, dass das Mess-Signal gedreht ist.
Denn das Mess-Signal verhält sich wie eine Sinus-Funktion, welche um $180\,°$
verschoben ist. Die Pulslänge beträgt $t_{\pi}$ = $\SI{5}{\micro\second}$ und es
kann schätzungsweise angenommen werden, dass das eingestrahlte
$B_1$-Feld stark genug ist.\\
Die im Unterkapitel \ref{sec:T1} bestimmte $T_1$-Zeit liegt bei
$T_1 = \SI{18.313\pm0.350}{\milli\second}$ und für den Exponenten $b$ der
Kohlrauschfunktion ergibt sich ein Wert von $b = \SI{1,037\pm0,024}{}$ und liegt
damit erwartungsgemäßg bei einem Wert von eins. Denn bei $b$=1 würde sich die longitudinale
Magnetisierung $M_{\text{z}}$, wie in der Theorie unter Gleichung \ref{eq:Mz}
beschrieben, verhalten. In Abbildung \ref{fig:T1} zeigt sich das typische Verhalten
der $T_1$-Relaxation. Nachdem die Magnetisierung $M$ in die transversale Ebene gekippt ist,
kann anfangs kaum longitudinale Magnetisierung $M_{\text{z}}$ gemessen werden.
Erst nachdem einige Millisekunden vergangen sind, kann bedingt durch die Spin-Gitter
Relaxation eine Magnetisierung $M_{\text{z}}$ gemessen werden. Nach der Erholungszeit
$t=T_1$ hat sich etwa $63\,\%$ der Magnetisierung aufgebaut. Schließlich ist nach
ungefähr $\SI{100}{\milli\second}$ die longitudinale Magnetisierung $M_{\text{z}}$
etwa auf den ursprünglichen Wert relaxiert ist.\\
Die $T_2$-Zeit wird im Unterkapitel \ref{sec:T2} bestimmt und ergibt eine Spin-Spin
Relaxationszeit von $\SI{0.259\pm0.008}{\milli\second}$. Damit sei die Bedingung:
$T_1 \gg T_2$ erfüllt.
Für den Exponenten ergibt sich ein Wert von $b = \SI{1,624\pm0,124}{}$ und weicht
damit ein wenig von der Theorie ab. Hier ist nach Gleichung \ref{eq:Mxy} ein Exponenten $b$ von
eins zu erwarten gewesen. In Abbildung \ref{fig:T2} zeigt sich das Verhalten der
Spin-Spin Relaxation. Sobald die Magnetisierung $M$ in die transversale Ebene gekippt
ist, zeigt sich betragsmäßig das Signal am stärksten. Nach der Erholungszeit
$t=T_2$ ist etwa $37\,\%$ der transversalen Magnetisierung $M_{\text{xy}}$
dephasiert. Aufgrund eines Problems mit dem Phasenzyklus wird hier der
Betrag des Mess-Signals: $|$Re$+$Im$|$ dargestellt. Damit ist das Problem leider
nur teilweise behoben, denn in Abbildung \ref{fig:T2} bei einer Evolutionszeit
von ungefähr $\SI{0,6}{\milli\second}$ zeigt sich eine erneuerte Ansteigung des
Mess-Singals. Die ist vermutlich auch der Grund für die Abweichung des Expontenten $b$.
Dieser fälschliche Anstieg ist in den meisten $T_2$-Messung so sehr ausgeprägt, dass
das anlegen einer Kohlrauschfunktion an die Messwerte nicht mehr gelingt (siehe Abbildung
\ref{fig:T2err} im Anhang).\\
Zunächst werden die beiden Abbildungen \ref{fig:cos} und \ref{fig:sin} aus den
stimulierten Echo Messungen verglichen. Der Kurvenverlauf der beiden Messungen
unterscheiden sich geringfügig. Bedingt durch die Dynamik bei der Umoriertierung
des Dimethylsulfon Moleküls nimmt die Kurve anfänglich ab, wobei beide Messungen
nach einer Mischzeit $t_{\text{m}}$ von etwa $\SI{1}{\milli\second}$ eine zunehmende
abfallende Steigung zeigen.
Im weiteren Kurvenverlauf zeigt sich nach ungefähr $\SI{4}{\milli\second}$,
bedingt durch die eintretende Dekorrelation, eine Änderung der Steigung (Knick).
Die Dynamik ist bei niedrigeren Temperaturen weniger, wodurch die Dekorrelations-Zeit
$\tau$ näher an die $T_1$/$T_{1,{\text{Q}}}$-Zeit rückt und deshalb der typische
zweistufen Zerfall, bei der sich anstelle des Knickes ein Plateau zeigt, kaum zu erkennen ist.
Zuletzt zeigen beide Kurven, aufgrund der Spin-Gitter Relaxation, ein
abgeschwächtes abfallendes Verhalten. Für denn Fall, dass die
Mischzeit $t_{\text{m}}$ gegen unendlich läuft, streben beide Funktionen gegen den
Fitparameter $S_0$, welche eine Ähnliche Amplitude besitzen ($S_{0_{\text{cos}}} = \SI{-0.062\pm0.003}{}$ und
$S_{0_{\text{sin}}} = \SI{-0.069\pm0.01}{}$).\\
Es fällt besonders auf, dass sich die beiden Korrelationszeiten $\tau$ (siehe
Tabelle \ref{tab:tau}) stark unterscheiden. Der Bedingung: $T_2 \ll \tau \ll T_1$
zufolge, ist die Korrelationszeit $\tau_{\text{cos}}$ der sinnvollere Wert.
Alledings haben die Messung der $T_2$-Zeit Probleme bereitet, weswegen man
nicht gewissenhaft auf den Wert vertrauen kann. Hingegen zeigt sich bei der
$sin$-$sin$-Messung ein Kurvenverlauf, der dem der Versuchanleitung \cite{Anleitung}[S.25]
ähnlich ist.\\
Die beiden Spin-Gitter-Relaxationszeiten sind ebenfalls im Tabelle \ref{tab:tau}
aufgeführt. Die als Fitparameter freigegebene $T_{1\text{,Q}}$-Zeit ist dabei mehr
als doppelt so lang, wie die $T_1$-Zeit.

\begin{table}
  \centering
  \caption{Korrelationszeit $\tau$ aus dem Unterkapitel \ref{sec:stecho}}
  \label{tab:tau}
  \sisetup{table-format=1.2}
  \begin{tabular}{S S S S}
    \toprule
    {$\tau_{\text{cos}}$ / ms} & {$T_1$ / ms} & {$\tau_{\text{sin}}$ / ms} & {$T_{1\text{,Q}}$ / ms}\\
    \midrule
    {1,838\pm0,093} & {18,313\pm0,350} & {1,854\pm0,064} & {31,385\pm2,661}\\
    \bottomrule
  \end{tabular}
\end{table}
\noindent
\\
Bei der Untersuchung der Temperaturabhängigkeit im Unterkapitel \ref{sec:tempabh}
konnte in Abbildung \ref{fig:tempabh} für die $cos$-$cos$- und $sin$-$sin$-Messungen
ein exponentieller Anstieg der Korrelationszeit $\tau$ mit kleiner werdenden Temperatur
beobachtet werden. Denn bei kleineren Temperaturen ist bedingt durch den
Boltzmann-Faktor $\exp{(-E/K_{\text{B}}T)}$, die Sprungrate kleiner als bei
höheren Temperaturen. Es zeigt sich in den $cos$-$cos$- und $sin$-$sin$-Messungen
eine annähernd übereinstimmende Aktivierungsenergie $E$.
Nachdem die Ausreißer in Abbildung \ref{fig:tempabh} aus der Auswertung ausgeschlossen
wurden, liegen die beiden Vorfaktoren $\tau_{0,\text{cos}}$ und $\tau_{0,\text{sin}}$
in der gleichen Größenordnung. (siehe Tabelle \ref{tab:tempabh}).
Eine mögliche Begründung für das abweichende Verhalten der fünf Werte könnte darin liegen,
dass die Korrelationszeit $\tau$ in der Größenordnung der gewählten Evolutionszeit
liegt und die Dynamik damit bereits während der Evolutionszeit statt gefunden hat.
Auffällig ist dabei, dass in der $sin$-$sin$-Messungen solche Ausreißer nicht zu
beobachten sind.\\
Die Spin-Gitter Relaxationszeit in Abbildung \ref{fig:tempabh} wird hingegen bei
kleineren Temperaturen kürzer, da auch diese dem Arrehenius-Gesetz folgt. Die
Spin-Spin-Relaxationszeit konnten leider nicht genauer untersucht werden (s.o.).

\begin{table}
  \centering
  \caption{Aktivierungsenergie $E$ und Vorfaktor $\tau_0$ aus Unterkapitel
  \ref{sec:tempabh}}
  \label{tab:tempabh}
  \sisetup{table-format=1.2}
  \begin{tabular}{S S S S}
    \toprule
    \multicolumn{2}{c}{cos-cos} & \multicolumn{2}{c}{sin-sin} \\
    {E / eV} & {$\tau_0$ / $\SI{1e-17}{\second}$} & {E / eV} & {$\tau_0$ / $\SI{1e-17}{\second}$} \\
    \midrule
    {0,84\pm0,01} & {3,39\pm0,06} & {0,83\pm0,01} & {5,16\pm0,06} \\
    \bottomrule
  \end{tabular}
\end{table}
\noindent
In dem Spektrum in Abbildung \ref{fig:spek} ist sind zwei Peaks zu erkenne, welche
durch den Übergang



Dublet-Struktur zu erkennen,
welches durch die beiden Deuteronen verursacht werden. Der Abstand zwischen den
Extremstellen beschreibt dabei ihre Kopplungskonstante, welche zum einen bei
$\SI{39,51}{\kilo\hertz}$ und zum anderen bei $\SI{44,66}{\kilo\hertz}$ liegt.
Im weiteren Auslauf der beiden benannten Extremstellen, sind kleine Extremstellen
zu beobachten. Da Kohlenstoff $^{12}C$ einen Kernspin $I$ von Null hat, rührt
dieser Anteil von dem Sauerstoff $O$ und dem Schwefel $S$.
