\section{Diskussion}
\label{sec:Diskussion}

Im ersten Unterkapitel \ref{sec:beo} sind phänomenologische Beobachtungen des Supraleiters
bei normaler Atmosphäre und bei Raumtemperatur
beschrieben. Dabei wird klar, dass der supraleitende Zustand schnell wieder vergeht,
wenn keine Arbeit in Form von Kühlung an ihm verrichtet wird. So arbeitet
beispielsweise ein supraleitendes Magnetlager reibungsfrei, braucht dagegen
aber viel Energie zur Kühlung, um den Zustand beizubehalten. Die Rotationsbewegungen
des Magneten in Abbildung \ref{fig:SL1} und \ref{fig:SL2} resultieren aus dem
Temperaturgradienten der umgebenden Luft und aus der Inhomogenität des Permanentmagneten.\\
Im Folgenden wird der temperaturabhängige supraleitende
Zustand durch zwei unterschied{\-}liche Messverfahren genauer untersucht. Als
erstes wird im Unterkapitel \ref{sec:TcOchse} mit dem bloßen Auge beobachtet,
bei welcher Temperatur der supraleitende Zustand des YBCO-Supraleiters vergeht.
In Abbildung \ref{fig:TcOchse} zeigen die ersten beiden Messungen 1 und 2 einen
ähnlichen Verlauf, allerdings eine um etwa $\SI{4}{\kelvin}$ unterschiedliche
kritische Temperatur $T^{\text{MCE}}_{\text{c}}$.
Das liegt vermutlich an der Messauflösung (eine Temperaturmessung pro Sekunde),
die dem großen Temperaturausschlag nach der kritischen Temperatur nicht mehr genügt.
Messung 3 zeigt besonders nach Erreichen der kritischen Temperatur $T^{\text{MCE}}_{\text{c}}$
ein anderes Verhalten als die beiden anderen Messungen. Der Temperaturanstieg
ist dort wesentlich geringer. Möglicherweise ist der Magnet, welcher noch eine
geringe Temperatur haben könnte, als der Supraleiter selbst, in die Nähe des
Temperatursensor gefallen und hemmt damit den raschen Temperaturanstieg. Aber auch
ein Eingriff nach dem Magnetabsenken in das Messsystem könnte den Verlauf dort
verfälscht haben.
Hier wäre es hilfreich, einen weiteren Temperatursensor mit einzubinden.
So ließe sich der systematische Fehler bei fehlerhafter Position des Sensors und
der Verfälschung durch einen Temperaturgradienten am Supraleiter minimieren.
Schließlich ergibt sich, nach Mittelung über drei Messungen, eine kritische Temperatur
von $\bar{T}^{\text{MCE}}_{\text{c}} = \SI{95.45\pm2.89}{\kelvin}$. Dieser Wert
liegt dem theoretischen Wert von maximal $\SI{93}{\kelvin}$ \cite[S. 62]{Hohenester}
sehr nah. \\
Beim zweiten Messverfahren im Unterkapitel \ref{sec:Tc4punkt} wird ein
Bi2223-Supraleiter in einem möglichst geschlossenen Messsystem mittels einer
4-Punkt-Messung auf seine kritische Temperatur $T^{\text{4PM}}_{\text{c}}$
untersucht. Dabei wird geschaut, bei welcher Temperatur der Widerstand von Null
verschieden wird. In Abbildung \ref{fig:Tc4PM} zeigt sich, dass die Messauflösung
bei diesem Verfahren gut genug ist. Die kritische Temperatur $T^{\text{4PM}}_{\text{c}}$ von
$\SI{117,4\pm2}{\kelvin}$ wird bei einem Widerstandsanstieg von Null auf $\SI{0.1}{\milli\ohm}$
abgelesen. Die theoretische kritische Temperatur liegt bei $\SI{110}{\kelvin}$
\cite[S. 64]{Hohenester} und damit um $\SI{7,4\pm2}{\kelvin}$ daneben. Eine Mittelung
über mehrere Messungen würde vermutlich ein genaueres Ergebnis liefern.\\
Desweiteren wird im Unterkapitel \ref{sec:mitB} der Einfluss eines extern
angelegten Magnetfeldes auf die kritische Temperatur $T^{\text{4PM}}_{\text{c}}$ untersucht.
Wie eben gezeigt, liegt die kritische Temperatur $T^{\text{4PM}}_{\text{c}}$ ohne
Magnetfeld bei $\SI{117,4\pm2}{\kelvin}$. Deshalb wird erwartet, dass sich
die kritische Temperatur $T^{\text{4PM}}_{\text{c}}$ mit größer werdendem Magnetabstand
$d$ diesem Wert annähert. Wie im Unterkapitel \ref{sec:phyHintergrunde} beschrieben,
bilden sich im Typ 2-Supraleiter normalleitende magnetische Flussschläuche aus. Diese
Flussschläuche spüren durch den im Supraleiter durchfließenden Strom $I$
eine Lorentzkraft $F_{\text{L}}$. In Hochtemperatursupraleitern, wie dieser hier,
können sich diese Flussschläuche wie in einer Flüssigkeit bewegen.
Diese Feldbewegung hat eine weitere Lorentzkraft $F_{\text{C}}$ zu Folge, welche
nach der Lenzschen Regel dem Durchlaufstrom I entgegenwirkt und damit einen
elektrischen Widerstand im Supraleiter verursacht. Je stärker nun das äußere
Magnetfeld $B_{\text{ex}}$, desto größer werden diese Flussschläuche und damit
die Lorentzkraft $F_{\text{C}}$ bzw. der elektrische Widerstand.
In Tabelle \ref{tab:Tc4PM1016} zeigt die kritische Temperatur
$T^{\text{4PM}}_{\text{c}}$ eine Änderung mit größer werdenden Abstand von
$\SI{0.3}{\kelvin}$. Leider lässt sich aus dieser zu geringen Änderung keine
Aussage treffen, da der systematische Fehler aller Messwerte bereits bei
$\SI{\pm2}{\kelvin}$ liegt. Um das Verhalten der kritischen Temperatur in
Abhängigkeit zur Magnetfeldstärke genauer untersuchen zu können, sollte die
kritische Temperatur für viele verschiedene Magnetfeldstärken bestimmt werden.\\
Im Weiteren konnte gezeigt werden, dass nicht nur die Temperatur eine entscheidende
Rolle für den supraleitenden Zustand spielt. Wird ein Supraleiter als Leiter
verwendet, stellt man fest, dass keine beliebig große Stromstärke durchfließen
kann. Ab einer kritischen Stromstärke $I_{\text{c}}$, welche im Unterkapitel \ref{sec:Ic}
für den Bi2223-Supraleiter abgeschätzt wird, verliert der Supraleiter seine
supraleitenden Eigenschaften. Dafür werden fünf Messungen, ohne die Störung durch ein
externes Magnetfeld, bei unterschiedlichen Durchflussstromstärken $I$ gemacht und mittels
einer linearen Extrapolation bis $\SI{77}{\kelvin}$ der kritische Strom $I_{\text{c}}$
abgeschätzt (siehe Abbildung \ref{fig:Ic}), wobei eine der fünf Messungen aus der
Auswertung entnommen wurde. Der Grund dafür ist im Unterkapitel \ref{sec:ohneB1}
zu finden. Der Schätzwert für die kritische Temperatur $I_{\text{c}}$ liegt bei
$\SI{11\pm4}{\ampere}$. Für Bi2223 können kritische Stromdichten $j_{\text{c}}$
von etwa $\SI{1.2e7}{\ampere\per\metre\squared}$ bei $\SI{5}{\kelvin}$
\cite[S. 345]{2223} beobachtet werden. Leider sind die Maße des Bi2223-Supraleiterstabs
nicht bekannt. Aber angenommen dieser hätte einen kreisförmigen Querschnitt, so
ergebe sich mit

\begin{equation*}
  \pi R_{\text{Bi2223-Stab}}^2 = \frac{I_{\text{c}}}{j_{\text{c}}}
  \iff
  R_{\text{Bi2223-Stab}} = \sqrt{\frac{I_{\text{c}}}{j_{\text{c}}\pi}}
\end{equation*}

\noindent
ein minimaler Radius $R_{\text{Bi2223-Stab}}$ von $\SI{0.5}{\milli\meter}$.
Der Stabradius wird größer gewesen sein, weshalb ein kritischer Stromwert
$I_{\text{c}}$ von etwa $\SI{11\pm4}{\ampere}$ bei $\SI{77}{\kelvin}$ durchaus
realistisch ist.\\
Als letzte 4-Punkt-Messung wird im Unterkapitel \ref{sec:mitB1} das Verhalten des
kritischen Stroms in Abhängigkeit eines externen Magnetfeldes untersucht.
Aufgrund von Schwierigkeiten beim Messen, auf die im Folgenden noch eingegangen wird,
werden hier nur Messungen für zwei unterschiedlich starke Durchlaufstromstärken $I$
untersucht. Unter diesen Umständen ergibt sich durch Extrapolation bis $\SI{77}{\kelvin}$
(siehe Abbildung \ref{fig:Ic2}) ein kritischer Strom $I_{\text{c}}$ von $\SI{2,42}{\ampere}$.
Wie im Unterkapitel \ref{para:typ2} beschrieben, befinden sich im Supraleiter zweiter Art
normalleitende Bereiche. Durch das von Außen angelegte Magnetfeld $B_{\text{ex}}$
wird der normalleitende Bereich vergrößert, sodass die kritische Stromstärke
$I_{\text{c}}$ circa 78$\,$\% kleiner als ohne Magnetfeld ist.\\
Besonders bei den letzten Messungen kam es wegen Feuchtigkeitsbildung durch das
Auf- und Abtauen des geschlossenen Messsystems zu Schwierigkeiten. Mit einem Heißluftfön
wurde probiert, dem entgegenzuwirken. Allerdings hat sich zum Ende hin so viel
Feuchtigkeit gesammelt, dass nur noch langes Warten bzw. Fönen genügt hätte. Dies
wird dem geschlossenen System zum Nachteil und erzeugt Sprungwerte wie z.B. in
Abbildung \ref{fig:Tc4PM}. Um diese vermeiden zu können, sollte zwischen den
Messungen genug Zeit abgewartet bzw. genügend gefönt werden.\\
Im letzten Unterkapitel \ref{sec:ind} wird der induzierte Strom am Bi2223-Supraleiter-Ring
untersucht. Mit dem Biot-Savart-Gesetz für kreisförmige Leiterschleifen ergibt
sich dabei ein Induktionsstrom von $\SI{18,98\pm0.51}{\ampere}$. Der Bi2223-Supraleiter-Ring
hat eine Wanddicke $d$ von $\SI{1,5}{\milli\meter}$ und damit schätzungsweise einen
Querschnitt von $A \approx \pi \frac{d^2}{4} = \SI{1,77e-6}{\meter\squared}$.
Wie oben erwähnt, ist die kritische Stromdichte $j_{\text{c}}$ für einen Bi2223-Supraleiter
etwa $\SI{1.2e7}{\ampere\per\metre\squared}$ bei $\SI{5}{\kelvin}$. Demnach
ergibt sich ein kritischer Strom $I_{\text{c}}$ für den Bi2223-Supraleiter-Ring
von $\SI{21,24}{\ampere}$. Ein Induktionsstrom von $\SI{18,98\pm0,51}{\ampere}$
wäre damit durchaus realistisch.
