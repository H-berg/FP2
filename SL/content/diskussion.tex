\section{Diskussion}
\label{sec:Diskussion}

Im ersten Unterkapitel \ref{sec:beo} sind phänomenologische Beobachtungen des Supraleiters
beschrieben. Dabei wird klar, dass der supraleitende Zustand schnell wieder vergeht,
wenn keine Arbeit, im Form von Kühlung, an ihm verrichtet wird. So arbeitet
beispielsweise ein supraleitendes Magnetlager reibungsfrei, braucht dagegen
aber viel Energie zur Kühlung, um den Zustand beizubehalten. Die Rotationsbewegungen
des Magneten in Abbildung \ref{fig:SL1} und \ref{fig:SL2} resultiert aus dem
Temperaturgradienten der umgebenden Luft und aus der Inhomogenität des Permanentmagneten.\\
Im folgendem wird der temperaturabhängige supraleitende
Zustand durch zwei unterschied{\-}lichen Messverfahren genauer untersucht. Als
erstes wird im Unterkapitel \ref{sec:TcOchse} mit dem bloßem Auge beobachtet,
bei welcher Temperatur der supraleitende Zustand des YBCO-Supraleiters vergeht.
In Abbildung \ref{fig:TcOchse} zeigen die ersten beiden Messungen 1 und 2 einen
ähnlichen Verlauf, allerdings eine um etwa $\SI{4}{\kelvin}$ unterschiedliche
kritsche Tempeatur $T^{\text{MCE}}_{\text{c}}$.
Das liegt vermutlich an der Messauflösung (eine Temperaturmessung pro Sekunde),
die dem großen Temperaturausschlag nach der kritschen Temperatur nicht mehr genügt.
Messung 3 zeigt besonders nach Erreichen der kritschen Temperatur $T^{\text{MCE}}_{\text{c}}$
ein anderes Verhalten als die beiden anderen Messungen. Der Temperaturanstieg
ist dort wesentlich geringer. Möglicherweise ist der Magnet, welcher noch eine
geringe Temperatur haben könnte, als der Supraleiter selbst, in die Nähe des
Temperatursensor gefallen und hemmt damit den raschen Temperaturanstieg. Aber auch
ein Eingriff nach dem Magnetabsenken in das Messsystem, könnte den Verlauf dort
verfälscht haben.
Hier wäre es hilfreich, einen weiteren Temperatursensor mit einzubinden.
So ließe sich der systematische Fehler bei fehlerhafter Position des Sensors und
der Verfälschung durch einen Temperaturgradient am Supraleiter minimieren.
Schließlich ergibt sich, nach Mittelung über drei Messungen, eine kritsche Temperatur
von $\bar{T}^{\text{MCE}}_{\text{c}} = \SI{95.45\pm2.89}{\kelvin}$. Dieser Wert
liegt dem theoretischem Wert von maximal $\SI{93}{\kelvin}$ \cite[S. 62]{Hohenester}
sehr nah. \\
Beim zweiten Messverfahren im Unterkapitel \ref{sec:Tc4punkt} wird ein
Bi2223-Supraleiter in einem möglichst geschlossenem Messsystem mittels einer
4-Punkt-Messung auf seine kritsche Temperatur $T^{\text{4PM}}_{\text{c}}$
untersucht. Dabei wird geschaut, bei welcher Temperatur der Widerstand von Null
verschieden wird. In Abbildung \ref{fig:Tc4PM} zeigt sich, dass die Messauflösung
bei diesem Verfahren gut genug ist. Die kritsche Temperatur $T^{\text{4PM}}_{\text{c}}$ von
$\SI{117,4\pm2}{\kelvin}$ wird bei einem Widerstandsanstieg von Null auf $\SI{0.1}{\milli\ohm}$
abgelesen. Die theoretische kritsche Temperatur liegt bei $\SI{110}{\kelvin}$
\cite[S. 64]{Hohenester} und damit um $\SI{7,4\pm2}{\kelvin}$ daneben. Eine Mittelung
über mehrere Messungen würde vermutlich ein genaueres Ergebnis liefern.\\
Desweiteren wird im Unterkapitel \ref{sec:mitB} der Einfluss eines extern
angelegtem Magnetfeld auf die kritsche Temperatur $T^{\text{4PM}}_{\text{c}}$ untersucht.
Wie eben gezeigt, liegt die kritsche Temperatur $T^{\text{4PM}}_{\text{c}}$ ohne
Magnetfeld bei $\SI{117,4\pm2}{\kelvin}$. Deshalb wird erwartet, dass sich
die kritsche Temperatur $T^{\text{4PM}}_{\text{c}}$ mit größer werdenden Magnetabstand
$d$ diesem Wert annähert. Gegen aller Erwartung zeigt sich denoch in Tabelle
\ref{tab:Tc4PM1016} eine temperaturabnahme mit größer werdenden Magnetabstand.
Selbst wenn probier wird, dem Verlauf des Widerstandanstiegs zu folgen, um die
Lücke in Abbildung \ref{fig:Tc4PM10} zu umgehen, ergibt sich in etwa eine kritsche
Temperatur $T^{\text{4PM}}_{\text{c}}$ von $\SI{104}{\kelvin}$. Dies würde an der
Tatsache nichts ändern. Dieses merkwürdige Verhalten lässt sich mit dieser Arbeit
nicht erklären und es müssten weiteren Messreihe zur Überprüfung gemacht werden. \\
Im weiteren konnte gezeigt werden, dass nicht nur die Temperatur eine entscheidende
Rolle für den supraleitenden Zustand spielt. Wird ein Supraleiter als Leiter
verwendet, stellt sich fest, dass keine beliebig große Stromstärke durchfließen
kann. Ab einer kritschen Stromstärke $I_{\text{c}}$, welche im Unterkapitel \ref{sec:Ic}
für den Bi2223-Supraleiter abegschätzt wird, verliert der Supraleiter seine
supraleitende Eigenschaften. Dafür werden fünf Messungen, ohne die Störung durch ein
externes Magnetfeld, bei unterschiedlichen Durchstromstärken $I$ gemacht und mittels
einer linearen Extrapolation bis $\SI{77}{\kelvin}$ der kritsche Strom $I_{\text{c}}$
abgeschätzt (siehe Abbildung \ref{fig:Ic}), wobei eine der fünf Messungen aus der
Auswertung entnommen wurde. Der Grund dafür ist im Unterkapitel \ref{sec:ohneB1}
zu finden. Der Schätzwert für die kritsche Temperatur $I_{\text{c}}$ liegt bei
$\SI{11.2\pm4,24}{\ampere}$. Für Bi2223-Supraleiter können kritsche Stromdichte $j_{\text{c}}$
von etwa $\SI{1.2e7}{\ampere\per\metre\squared}$ bei $\SI{5}{\kelvin}$
\cite[S. 345]{2223} beobachtet werden. Leider sind die Maße des Bi2223-Supraleiterstabs
nicht bekannt. Aber angenommen dieser hätte einen kreisförmigen Querschnitt, so
ergebe sich mit

\begin{equation*}
  \pi R_{\text{Bi2223-Stab}}^2 = \frac{I}{A}
  \iff
  R_{\text{Bi2223-Stab}} = \sqrt{\frac{\pi}{j_{\text{c}}\pi}}
\end{equation*}

\noindent
ein minimalen Radius $R_{\text{Bi2223-Stab}}$ von $\SI{0.55}{\milli\meter}$.
Der Stabradius wird größer gewesen sein, weshalb ein kritscher Stromwert
$I_{\text{c}}$ von etwa $\SI{11.2\pm4,24}{\ampere}$ bei $\SI{77}{\kelvin}$ durchaus
realistisch ist.\\
Als letzte 4-Punkt-Messung wird im Unterkapitel \ref{sec:mitB1} das Verhalten des
kritischen Stroms in Abhängigkeit eines externen Magnetfeld untersucht.
Aufgrund von Schwierigkeiten beim Messen, auf die im folgendem noch eingegangen wird,
werden hier nur Messungen für zwei unterschiedlich starke Durchlaufstromstärken $I$
untersucht. Unter diesen Umständen ergibt sich durch Extrapolation bis $\SI{77}{\kelvin}$
(siehe Abbildung \ref{fig:Ic2}) ein kritschen Strom $I_{\text{c}}$ von $\SI{2,42}{\ampere}$.
Wie im Unterkapitel (\ref{para:typ2}) beschrieben, befinden sich im Supraleiter zweiter Art
normalleitende Bereiche. Durch das von Außen angelegte Magnetfeld $B_{\text{ex}}$
wird der normalleitende Bereich vergrößert, sodass die kritsche Stromstärke
$I_{\text{c}}$ circa 78$\,$\% kleiner als ohne Magnetfeld.\\
Besonders bei den letzten Messungen kam es, wegen Feuchtigkeitsbildung durch das
auf und abtauen des geschlossenem Messsystem, zu Schwierigkeiten. Mit einem Heißluftfön
wurde probiert dem entgegenzuwirken. Allerdings hat sich zum Ende hin so viel
Feuchtigkeit gesammelt, dass nur noch langes warten bzw. fönen genügt hätte. Dies
wird dem geschlossenem System zum Nachteil und erzeugt Sprungwerte wie z.B. in
Abbildung \ref{fig:Tc4PM}. Um diese vermeiden zu können, sollte zwischen den
Messungen genug Zeit abgewartet bzw. genügend gefönt werden.\\
Im letzten Unterkapitel \ref{sec:ind} wird der induzierte Strom am Bi2223-Supraleiter-Ring
untersucht. Mit dem Biot-Savart-Gesetz für kreisförmige Leiterschleifen ergibt
sich dabei ein Induktionsstrom von $\SI{18,98\pm0.51}{\ampere}$. Der Bi2223-Supraleiter-Ring
hat eine Wanddicke $d$ von $\SI{1,5}{\milli\meter}$ und damit schätzungsweise einen
Querschnitt von $A \approx \pi \frac{d^2}{4} = \SI{1,77e-6}{\meter\squared}$.
Wie oben erwähnt, ist die kritsche Stromdichte $j_{\text{c}}$ für einen Bi2223-Supraleiter
bei etwa $\SI{1.2e7}{\ampere\per\metre\squared}$ bei $\SI{5}{\kelvin}$. Demnach
ergibt sich ein kritscher Strom $I_{\text{c}}$ für den Bi2223-Supraleiter-Ring
von $\SI{21,24}{\ampere}$. Ein Induktionsstrom von $\SI{18,98\pm0,51}{\ampere}$
wäre damit durchaus realistisch.
