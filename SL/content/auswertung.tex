\section{Auswertung}
\label{sec:Auswertung}

\subsection{Phänomenologische Beobachtungen des Supraleiters}
\label{sec:beo}
Zu Beginn des Versuches werden die gemachten Beobachtungen der verschiedenen
Magnet-Supraleiter-Anordnungen geschildert. In den drei folgenden Abbildungen
\ref{fig:SL1}, \ref{fig:SL2} und \ref{fig:SL3} ist gut zu erkennen,
wie der Magnet, bedingt durch den Meißner-Ochsenfeld-Effekt, eigenstabil über dem
Supraleiter schwebt. Dabei rotiert in Abbildung \ref{fig:SL1} ein kleiner Magnet
\#M1 in einer schrägen Position über dem SL \#2 Supraleiter um seine Achse hin und zurück.
Hingegen schwebt der größere Magnet \#M2 in Abbildung \ref{fig:SL2} parallel
zum SL \#3 Supraleiter und ändert dabei nach einigen Rotationen seine Rotationsrichtung.

\begin{figure}[H]
\centering
	\begin{subfigure}[t]{0.45\textwidth}
    \includegraphics[width=\textwidth]{Auswertung/SL_1.jpg}
    \caption{Der SL \#2 Supraleiter zusammen mit dem \#M1 Permanentmagneten auf
		einem Styroporbehältnis mit Flüssigstickstoff.}
    \label{fig:SL1}
	\end{subfigure}
	~
	\begin{subfigure}[t]{0.45\textwidth}
    \includegraphics[width=1.06\textwidth]{Auswertung/SL_2.jpg}
    \caption{Der SL \#3 Supraleiter zusammen mit dem \#M2 Permanentmagneten auf
		einem Styroporbehältnis mit Flüssigstickstoff.}
    \label{fig:SL2}
	\end{subfigure}
\end{figure}

\noindent
Besonders attraktiv sind diese Eigenschaft in der Technik,
da so eine berührungslose Kraftübertragung realisiert werden kann. So lassen sich
beispielsweise supraleitende Magnetlager wie in Abbildung \ref{fig:SL3} realisieren.
In Abbildung \ref{fig:SL3} ist dazu ein Pinning-Stab zusammen mit dem \#M2
Permanentmagneten zu sehen. Diese Konfiguration schwebt über dem SL \# 3 Supraleiter.
Wird auf den Pinning-Stab samt \#M2 Magnet ein Impuls übertragen, so rotiert die Konfiguration
stabil über dem SL \#3 Supraleiter. Wird die Anordnung aus Abbildung \ref{fig:SL3} nun
am Pinning-Stab angehoben, so bleibt die Konstellation bestehen. Der SL \# 3 Supraleiter
schwebt nun unter dem Magneten, dieser Effekt wird der Suspension zugeschrieben.
Sind die Pinning-Kräfte stark genug, so lassen sich einfache Rotationsbewegungen
des Supraleiters in sämtliche Raumrichtungen ermöglichen (siehe Abbildung \ref{fig:SL4}).

\begin{figure}[H]
\centering
	\begin{subfigure}[t]{0.45\textwidth}
    \includegraphics[width=\textwidth]{Auswertung/SL_3.jpg}
    \caption{Magnetlager aus dem SL \#3 Supraleiter zusammen mit einem
    \#M2 Permanentmagneten und einem Pinning-Stab auf einem Styroporbehältnis mit
    Flüssigstickstoff.}
    \label{fig:SL3}
	\end{subfigure}
	~
	\begin{subfigure}[t]{0.45\textwidth}
    \includegraphics[width=\textwidth]{Auswertung/SL_4.jpg}
    \caption{Magnetlager aus dem SL \#3 Supraleiter zusammen mit einem
    \#M2 Permanentmagneten und einem Pinning-Stab in einer Schräglage angehoben.}
    \label{fig:SL4}
	\end{subfigure}
\end{figure}

\subsection{Bestimmung der kritischen Temperatur $T_{\text{c}}$ via Meißner-Ochsenfeld-Effekt}
\label{sec:TcOchse}

Um die Spannung $U_{\text{TS,Si}}$ am Silizium-Temperatursensor in Temperatur $T$
umzurechnen, wird ein Polynom vierten Grades genutzt:

\begin{equation*}
  T(U_{\text{TS,Si}}) = a_0 + a_1 \cdot U_{\text{TS,Si}} + a_2 \cdot U_{\text{TS,Si}}^2
  + a_3 \cdot U_{\text{TS,Si}}^3 + a_4 \cdot U_{\text{TS,Si}}^4,
  \label{FA1}
\end{equation*}

\noindent
dabei sind die Kalibrierungsdaten $a_{i}$ mit $i \in$ {0,..,4} dem Datenblatt
zu entnehmen. Das dabei resultierende Verhältnis zwischen Temperatur $T$ und der
Messzeit $t$ ist in Abbildung \ref{fig:TcOchse} dargestellt. Dabei wird die anfängliche
Messzeit abgeschnitten, bei der keine Temperaturänderung stattfindet und ein
neuer zeitlicher Nullpunkt gesetzt. So lässt sich das Verhalten bei der
Temperaturerhöhung besser untersuchen. Der Zeitpunkt an dem der \#M1 Magnet wieder
auf dem Supraleiter aufliegt, ist jeweils mit einem gelben Kreis gekennzeichnet.
Alle drei Messungen zeigen anfangs einen starken Temperaturanstieg. Etwa $\SI{4}{\second}$
nachdem die jeweilige kritische Temperatur $T^{\text{MCE}}_{\text{c}}$ erreicht ist, zeigen
Messung 1 und 2 einen schwächer werdenden Temperaturanstieg, bis dieser sich nach
weiteren $\SI{5}{\second}$ annähernd linear verhält. Messung 3 hingegen zeigt
sofort, nachdem die kritische Temperatur $T^{\text{MCE}}_{\text{c}}$ erreicht ist, ein linearen
Zuwachs der Temperatur.\\
Die sich dabei ergebende kritische Temperatur $T^{\text{MCE}}_{\text{c}}$ der einzelnen
Messungen ist in Tabelle \ref{tab:TcOchse} aufgelistet. Die Abschätzung des
systematischen Fehlers von $\SI{5}{\kelvin}$ ergibt sich teils daraus, dass das
Absenken des \#M1 Magnets nicht abrupt passiert, sondern in einem gewissen Messbereich.
Auch spielt die Position des Temperatursensors eine Rolle. Denn dieser liegt
möglicherweise nicht optimal auf dem SL \#2 Supraleiter auf. Ein Temperaturgradient
am SL \#2 Supraleiter, welcher die Messung verfälschen würde, ist ebenfalls nicht
auszuschließen. Da der Temperatursensor im Werk kalibriert wurde, wird der Fehler
beim Messen eher als gering eingeschätzt.
Durch Mittelung der kritischen Temperaturen $T^{\text{MCE}}_{\text{c}}$ der einzelnen Messungen
ergibt sich $\bar{T}^{MCE}_{\text{c}}$, mit dem Fehler $\Delta \bar{T}^{MCE}_{\text{c}} =
\Delta T^{MCE}_{\text{c}}/\sqrt{3}$.



\begin{figure}[H]
    \centering
    \includegraphics[width=0.8\textwidth]{Auswertung/T_krit_Si/T_krit.pdf}
    \caption{Bestimmung der kritischen Temperatur anhand von drei Messungen mittels
    des Meißner-Ochsenfeld-Effekts. Die gelben Kreise kennzeichnen den Zeitpunkt,
    an dem der \#M1 Magnet wieder auf dem SL \#2 Supraleiter aufliegt. Es wird
		allen Messwerten ein systematischer Fehler von $\SI{\pm5}{\kelvin}$
		zugeschrieben.}
    \label{fig:TcOchse}
\end{figure}

\begin{table}
  \centering
  \caption{Kritische Temperatur $T^{\text{MCE}}_{\text{c}}$ für drei Messungen.}
  \label{tab:TcOchse}
  \sisetup{table-format=1.2}
  \begin{tabular}{S S S | S}
    \toprule
    \multicolumn{3}{c}{$T^{\text{MCE}}_{\text{c}}$ / K} & {$\bar{T}^{\text{MCE}}_{\text{c}}$ / K} \\
    {Messung 1} & {Messung 2} & {Messung 3} & {Mittel} \\
    \midrule
    {92,9\pm5} & {96,5\pm5} & {96,9\pm5} & {95,5\pm2,9} \\
    \bottomrule
  \end{tabular}
\end{table}


\subsection{Bestimmung der kritischen Temperatur $T_{\text{c}}$ via 4-Punkt-Messung}
\label{sec:Tc4punkt}
Im Folgenden wird im Unterkapitel \ref{sec:ohneB} der SL \#1 Supraleiter ohne
Störung eines Magnetfeldes untersucht und damit die kritische Temperatur
$T^{\text{4PM}}_{\text{c}}$ bestimmt. Dann wird im Unterkapitel \ref{sec:mitB}
die Messung mit dem \#M3 Magneten für zwei verschiedene Abstände wiederholt.\\
Der Platin-Sensor erlaubt es die Temperatur am SL \#1 Supraleiter zu bestimmen.
Dazu wird der gemessene Widerstand $R_{\text{TS,Pt}}$ durch ein Polynom dritten
Grades,

\begin{equation*}
  T(R_{\text{TS,Pt}}) = a_0 + a_1 \cdot R_{\text{TS,Pt}} + a_2 \cdot R_{\text{TS,Pt}}^2
  + a_3 \cdot R_{\text{TS,Pt}}^3
  \label{FA2}
\end{equation*}

\noindent
in die Temperatur $T$ umgerechnet, wobei $a_{i}$ mit $i \in$ {0,..,3} dem
Datenblatt des Platin-Sensors entnommen wird. Der SL \#1 Supraleiter sowie
der Platin-Sensor befinden sich in einem geschlossenen Plexiglasgehäuse, was
die Messung vor äußeren Einflüssen schützt. Allerdings kann sich beim Auftauen
so besser Flüssigkeit ansammeln, welche das Messergebnis verfälschen kann.
Der Platin-Sensor ist vom Werk aus kalibriert und arbeitet damit sehr genau.
Ein Temperaturgradient am Supraleiter zum Platin-Sensor ist auch hier nicht
auszuschließen. Der systematische Fehler wird damit auf $\SI{\pm2}{\kelvin}$
geschätzt. Zusätzlich wird ein Ablesefehler bei der Bestimmung der kritischen
Temperatur $T^{\text{4PM}}_{\text{c}}$ dazugeschätzt. Dieser bildet sich durch
die Mittelung von vier Temperaturänderungen $\Delta T^{\text{4PM}}_{i,i-1}
= |T^{\text{4PM}}_{i}-T^{\text{4PM}}_{i-1}|$
um die kritische Temperatur $T^{\text{4PM}}_{\text{c}}$:

\begin{equation*}
	\Delta \bar{T}^{\text{4PM}}_{\text{c}} = \frac{
																								 \Delta T^{\text{4PM}}_{\text{c-1,c-2}}
																								+\Delta T^{\text{4PM}}_{\text{c,c-1}}
																								+\Delta T^{\text{4PM}}_{\text{c+1,c}}
																								+\Delta T^{\text{4PM}}_{\text{c+2,c+1}}
																								}
																								{4}
\label{FA3}
\end{equation*}

\noindent

\subsubsection{Ohne Magnet}
\label{sec:ohneB}

Das temperaturabhängige Verhalten des SL \#1 Supraleiter-Widerstandes $R_{\text{SL}}$,
ohne Störung durch einen Magnetfeld, bei einem Durchlaufstrom
$I$ von $\SI{0.6}{\ampere}$ ist in Abbildung \ref{fig:Tc4PM} dargestellt. Der
SL \#1 ist zwischen $\SI{76\pm2}{\kelvin}$ und etwa $\SI{117\pm2}{\kelvin}$ supraleitend,
denn hier zeigt sich ein Widerstandswert von Null. Fehlerhaft gemessene negative
Widerstandswerte sind größtenteils
aus dem Ausschnitt skaliert, weshalb Lücken, wie z.B. zwischen $\SI{80\pm2}{\kelvin}$
und $\SI{90\pm2}{\kelvin}$, entstehen. Diese und weitere Sprungwerte (beispielsweise
zwischen $\SI{115\pm2}{\kelvin}$ und etwa $\SI{116\pm2}{\kelvin}$) sind Messfehler, welche
sich größtenteils nicht im interessanten Bereich aufhalten. Der interessante Bereich
ist der, an dem der Widerstand $R_{\text{SL}}$ eine stetige Zunahme erfährt.
Bei einer Temperatur von etwa $\SI{117,4\pm2,6}{\kelvin}$ zeigt sich der erste
Widerstandsanstieg von $\SI{0,1}{\milli\ohm}$. Danach ist der Widerstandszuwachs
sehr stark, bis dieser sich bei etwa $\SI{125\pm2}{\kelvin}$ annähernd linear
verhält.


\begin{figure}[H]
    \centering
    \includegraphics[width=0.8\textwidth]{Auswertung/T_krit_Pt/R_T.pdf}
    \caption{Bestimmung der kritischen Temperatur des SL \#1 Supraleiters ohne Magneten
		anhand einer 4-Punkt-Messung. Die Temperatur $T$ wird mit einem Platin-Sensor
		ermittelt, wobei der Widerstand $R_{\text{SL}}$ bei einem Strom $I$ von $\SI{0.6}{\ampere}$
		gemessen wird. Die gelbe Markierung kennzeichnet die kritische Temperatur
		$T^{\text{4PM}}_{\text{c}}$. Es wird allen Messwerten ein systematischer
		Fehler von $\SI{\pm2}{\kelvin}$	zugeschrieben.}
    \label{fig:Tc4PM}
\end{figure}

\subsubsection{Mit Magnet}
\label{sec:mitB}
Die Messung für einen Magnetabstand von $\SI{10}{\milli\meter}$ ist in Abbildung
\ref{fig:Tc4PM10} und die in einem Abstand von $\SI{16}{\milli\meter}$ ist
in Abbildung \ref{fig:Tc4PM16} dargestellt. In beiden Fällen ist der Supraleiter
von $\SI{78\pm2}{\kelvin}$ bis zur kritischen Temperatur $T^{\text{4PM}}_{\text{c}}$ supraleitend.
Zuerst werden die beiden Abbildungen \ref{fig:Tc4PM10} und \ref{fig:Tc4PM16}
miteinander verglichen. Beide Messungen zeigen nach der kritischen
Temperatur $T^{\text{4PM}}_{\text{c}}$ einen ähnlich starken Widerstandsanstieg.
Werden beide Messungen (mit Magnet) mit der Messung aus Abbildung \ref{fig:Tc4PM}
(ohne Magnet) verglichen, fällt auf, dass der Widerstandsanstieg weniger stark ist.
Weiter fällt auf, dass die Messung aus Abbildung \ref{fig:Tc4PM10}
deutlich mehr Sprungwerte aufweist, als die Messung aus Abbildung \ref{fig:Tc4PM16}.
Einige Sprungwerte sind in Abbildung \ref{fig:Tc4PM10} leider genau dort, wo der
Widerstandsanstieg anfängt. Um die kritische Temperatur aus beiden Messungen
bei einem äquivalenten Widerstand von $\SI{0.05}{\milli\ohm}$ bestimmen zu können,
wird bei der Messung bei $\SI{10}{\milli\meter}$ Abstand eine Interpolation
durchgeführt.
\newpage
\noindent
Dabei werden nur die Messwerte, welche in Abbildung \ref{fig:Tc4PM10}
mit einem Kreuz markiert sind, für folgende Fit-Funktion

\begin{equation*}
	R_{\text{SL}}(T) = \frac{A_1 - A_2}{1+\exp\bigl( \frac{T-T_0}{dT} \bigr)} + A_2
\end{equation*}

\noindent
berücksichtigt. Nach der ersten Regression mittels $\textit{Python 3.7.6}$
ergibt sich ein $\textit{Offset}$ von $A_1 = \SI{-0.022}{\milli\ohm}$, weswegen
alle Messwerte um diesen Wert angehoben werden, sodass sich im supraleitenden
Zustand (T < $T^{\text{4PM}}_{\text{c}}$) ein Widerstand von Null ergibt. Eine
zweite Regression ergibt dann folgende Fit-Parameter

\begin{equation*}
	A_1 = \SI{0\pm0.03}{\milli\ohm}
	\quad
	A_2 = \SI{8.4\pm0.4}{\milli\ohm}
	\quad
	T_0 = \SI{3.7\pm0.1}{\kelvin}
	\quad
	dT	= \SI{119.2\pm0.1}{\kelvin}.
\end{equation*}

\noindent
In Tabelle \ref{tab:Tc4PM1016} sind die jeweilige kritischen Temperaturen
$T^{\text{4PM}}_{\text{c}}$ und die gemessene Magnetfeldstärke $B_{\text{ex}}$
aufgelistet.


%Gegen aller Erwartung, nimmt die $T^{\text{4PM}}_{\text{c}}$
%bei einer größeren Magnetfeldstärke zu.


\begin{figure}[H]
    \centering
    \includegraphics[width=0.8\textwidth]{Auswertung/T_krit_Pt/R_T_10mm.pdf}
    \caption{Bestimmung der kritischen Temperatur des SL \#1 Supraleiters mit dem
		\#M3 Magneten im Abstand von $\SI{10}{\milli\meter}$ anhand einer 4-Punkt-Messung.
		Die Temperatur $T$ wird mit einem Platin-Sensor	ermittelt, wobei der Widerstand
		$R_{\text{SL}}$ bei einem Strom $I$ von $\SI{0.6}{\ampere}$ gemessen wird.
		Aufgrund des sprunghaften Verhaltens, wird eine Interpolation durch die
		Funktion $R_{\text{SL}}(T) = \frac{A_1 - A_2}{1+\exp\bigl( \frac{T-T_0}{dT} \bigr)} + A_2$
		durchgeführt. Dabei werden nun die Messwerte berücksichtigt, die mit einem Kreuz
		markiert sind und damit in dem rot gestrichelten Schlauch liegen.
		Die gelbe Markierung kennzeichnet die kritische Temperatur $T^{\text{4PM}}_{\text{c}}$.
		Es wird allen Messwerten ein systematischer Fehler von $\SI{\pm2}{\kelvin}$
		zugeschrieben.}
    \label{fig:Tc4PM10}
\end{figure}

\begin{figure}[H]
    \centering
    \includegraphics[width=0.8\textwidth]{Auswertung/T_krit_Pt/R_T_16mm.pdf}
    \caption{Bestimmung der kritischen Temperatur des SL \#1 Supraleiters mit dem
		\#M3 Magneten im Abstand von $\SI{16}{\milli\meter}$ anhand einer 4-Punkt-Messung.
		Die Temperatur $T$ wird mit einem Platin-Sensor	ermittelt, wobei der Widerstand
		$R_{\text{SL}}$ bei einem Strom $I$ von $\SI{0.6}{\ampere}$ gemessen wird.
		Die gelbe Markierung kennzeichnet die kritische Temperatur	$T^{\text{4PM}}_{\text{c}}$.
		Es wird allen Messwerten ein systematischer Fehler von $\SI{\pm2}{\kelvin}$
		zugeschrieben.}
    \label{fig:Tc4PM16}
\end{figure}

\begin{table}
  \centering
  \caption{Kritische Temperatur $T^{\text{MCE}}_{\text{c}}$ für zwei
	unterschiedliche Magnetabstände.}
  \label{tab:Tc4PM1016}
  \sisetup{table-format=1.2}
  \begin{tabular}{S | S S}
    \toprule
    {Magnetabstand} & {kritische Temperatur} & {Magnetfeldstärke} \\
    {d / mm} & {$T^{\text{MCE}}_{\text{c}}$ / K } & {$B_{\text{ex}}$ / mT} \\
    \midrule
    {10} & {99,8\pm2}	&	{142,7}	\\
		{16} & {100,1\pm2,9}	&	{23,5}	\\
    \bottomrule
  \end{tabular}
\end{table}

\subsection{Abschätzung der kritischen Stromstärke $I_{\text{c}}$}
\label{sec:Ic}
Im folgenden Unterkapitel \ref{sec:ohneB1} wird die kritische Temperatur
$T^{\text{MCE}}_{\text{c}}$ bei einer 4-Punkt-Messung in Abhängigkeit von der
Durchlaufstromstärke $I$ untersucht. Die dabei entstehenden Fehler werden analog
zum Unterkapitel \ref{sec:Tc4punkt} abgeschätzt. Im zweiten Unterkapitel
\ref{sec:mitB1} wird die Messung mit Magnetfeld durchgeführt.

\subsubsection{Ohne Magnet}
\label{sec:ohneB1}
Eine 4-Punkt-Messung ohne Magnetfeld wird für jeweils fünf unterschiedliche
Stromstärken ($I=\SI{0,2}{\ampere}; \SI{0,4}{\ampere}; \SI{0,6}{\ampere};
\SI{0,8}{\ampere}; \SI{1,0}{\ampere}$) durchgeführt. Die kritische Temperatur
$T^{\text{MCE}}_{\text{c}}$ wird möglichst bei einem Widerstandsanstieg auf
$\SI{0.5}{\ampere}$ abgelesen (siehe Anhang \ref{sec:AnhangTc}) und
in Tabelle \ref{tab:Ic} aufgelistet. Mit Ausnahme
der kritischen Temperatur $T^{\text{MCE}}_{\text{c}}$ bei einer Stromstärke von
$\SI{0.6}{\ampere}$, ist die Tendenz, dass die kritische Temperatur
$T^{\text{MCE}}_{\text{c}}$ mit der Stromstärke $I$ abnimmt. Ein Blick in die Messung
bei einer Stromstärke von $\SI{0,6}{\ampere}$ (Abbildung \ref{fig:Ic1.3} im Anhang) zeigt
eine Lücke genau im interessanten Bereich des Widerstandanstiegs
und damit, im Vergleich zu den anderen Messungen, einen doppelt so großen
Widerstand $R_{\text{SL}}$ von $\SI{0.1}{\milli\ohm}$. Aus diesem Grund wird dieser
Wert in der folgenden Auswertung nicht mehr berücksichtigt.

\begin{table}
  \centering
  \caption{Kritische Temperatur $T^{\text{MCE}}_{\text{c}}$ für fünf unterschiedliche
	Durchlaufstromstärken.}
  \label{tab:Ic}
  \sisetup{table-format=1.2}
  \begin{tabular}{S | S}
    \toprule
    {Stromstärke} & {kritische Temperatur} \\
    {I / A} & {$T^{\text{MCE}}_{\text{c}}$ / K }  \\
    \midrule
    {0,2} & {116,83\pm2,54}	\\
		{0,4} & {116,61\pm2,55}	\\
		{0,6} & {117,36\pm2,60}	\\
		{0,8} & {114,83\pm2,61}	\\
		{1,0} & {114,17\pm2,55}	\\
    \bottomrule
  \end{tabular}
\end{table}

\noindent
Um nun die kritische Stromstärke $I_{\text{c}}$ bestimmen zu können, wird mittels
$\textit{Python 3.7.6}$ eine lineare Extrapolation bis $\SI{77}{\kelvin}$ gemacht,
welche in Abbildung \ref{fig:Ic} zu sehen ist. Für die Ausgleichsgerade
$I(T)=m\cdot T + b$
ergeben sich folgende Parameter

\begin{equation*}
	m = \SI{-0,27\pm0,03}{\ampere\per\kelvin}
	\qquad
	b = \SI{32,34\pm3,53}{\ampere}
	\label{AF4}
\end{equation*}

\noindent
und damit eine Abschätzung für den kritischer Strom $I_{\text{c}}$ von
$\SI{11\pm4}{\ampere}$. Der Fehler ergibt sich hierbei nach der Gaußschen
Fehlerfortpflanzung: $\Delta I(T) = \sqrt{(T\cdot \Delta m)^2+(\Delta b)^2}$

\begin{figure}[H]
    \centering
    \includegraphics[width=0.8\textwidth]{Auswertung/I_krit_Pt/I_krit.pdf}
    \caption{Bestimmung der krituschen Stromstärke $I_{\text{c}}$ des SL \#1 Supraleiters ohne
		Magnet anhand einer 4-Punkt-Messung. Die kritischen Temperaturen $T^{\text{MCE}}_{\text{c}}$
		bei der dazugehörigen Stromstärke $I$ sind der Tabelle \ref{tab:Ic} zu entnehmen.
		Die gelbe Markierung kennzeichnet die Abschätzung des kritischen Stroms $I_{\text{c}}$
		und die rote Markierung kennzeichnet den aus der Auswertung unberücksichtigten Wert.}
\label{fig:Ic}
\end{figure}

\subsubsection{Mit Magnet}
\label{sec:mitB1}
Nun erfährt der SL \#1 Supraleiter eine Störung durch das starke Magnetfeld des
\#M3 Magneten. Wegen Messproblemen wird der kritische Strom $I_{\text{c}}$ nur für zwei
Durchlaufstromstärken von $\SI{0,6}{\ampere}$ und $\SI{0,8}{\ampere}$ untersucht.
Dazu wird von jeder Messung
die kritische Temperatur $T^{\text{MCE}}_{\text{c}}$ bei einem Widerstand von etwa
0,5-$\SI{0,6}{\ampere}$ abgelesen (siehe Anhang \ref{sec:AnhangTc}) und der
jeweilige Wert in Tabelle \ref{tab:Ic2} eingetragen.

\begin{table}
  \centering
  \caption{Kritische Temperatur $T^{\text{MCE}}_{\text{c}}$ für fünf unterschiedliche
	Durchlaufstromstärken mit Magnetfeld des \#M3 Magneten.}
  \label{tab:Ic2}
  \sisetup{table-format=1.2}
  \begin{tabular}{S | S}
    \toprule
    {Stromstärke} & {kritische Temperatur} \\
    {I / A} & {$T^{\text{MCE}}_{\text{c}}$ / K }  \\
    \midrule
		{0,6} & {107,33\pm3,51}	\\
		{0,8} & {104,00\pm2,85}	\\
    \bottomrule
  \end{tabular}
\end{table}

\noindent
Die kritische Stromstärke $I_{\text{c}}$ ergibt sich erneut durch eine Extrapolation
bis $\SI{77}{\kelvin}$ mittels $\textit{Python 3.7.6}$, welche in Abbildung
\ref{fig:Ic2} dargestellt ist. Die Parameter der Ausgleichsgeraden
$I(T)=m\cdot T + b$ haben dabei folgende Werte

\begin{equation*}
	m = \SI{-0.06}{\ampere\per\kelvin}
	\qquad
	b = \SI{7.05}{\ampere}
	\label{AF5}
\end{equation*}

\noindent
und der kritische Strom $I_{\text{c}}$ damit einen Schätzwert von $\SI{2.4}{\ampere}$.

\begin{figure}[H]
    \centering
    \includegraphics[width=0.8\textwidth]{Auswertung/I_krit_Pt_b/I_krit.pdf}
    \caption{Bestimmung der kritischen Stromstärke $I_{\text{c}}$ des SL \#1
		Supraleiters mit Störung durch den \#M3 Magnet anhand einer 4-Punkt-Messung.
		Die kritischen Temperaturen $T^{\text{MCE}}_{\text{c}}$
		bei der dazugehörigen Stromstärke $I$ sind der Tabelle \ref{tab:Ic} zu entnehmen.
		Die gelbe Markierung kennzeichnet die Abschätzung des kritischen Stroms $I_{\text{c}}$.}
\label{fig:Ic2}
\end{figure}

\subsection{Abschätzung des induzierten Stroms $I_{\text{ind}}$}
\label{sec:ind}
Zuletzt wird der mit dem \#M3 Magneten induzierte Strom $I_{\text{ind}}$ am SL
\#4 Supraleiter-Ring abgeschätzt. Dazu misst die Hall-Sonde in einem Abstand von
$z=0$ zum Mittelpunkt des Rings ein induziertes Magnetfeld $B_{\text{ind}}$ von
$\SI{1,59}{\milli\tesla}$. Das Biot-Savart-Gesetz für kreisförmige Stromschleifen:

\begin{equation*}
	I_{\text{ind}}(z,R,B_{\text{ind}}) = \frac{2B_{\text{ind}}(z^2+R^2)^{3/2}}{\mu_0R^2}
\label{AF6}
\end{equation*}

\noindent
erlaubt es nun, den induzierten Strom $I_{\text{ind}}$ abschätzen, wobei der
Radius $R$ des SL \#4 Supraleiter-Rings $\SI{7,5}{\milli\meter}$ beträgt. Der
induzierte Strom lässt sich damit auf einen Wert von $\SI{18.98\pm0.51}{\ampere}$
abschätzen. Wegen einer möglichen nicht idealen Messposition in $z$-Richtung,
wird dem $I_{\text{ind}}$-Wert ein Fehler, welcher sich aus dem absoluten Fehler
$|I_{\text{ind}}(z=0,R,B_{\text{ind}}) - I_{\text{ind}}(z=\SI{\pm0.1}{\milli\meter},R,B_{\text{ind}})|$
ergibt, zugeschrieben.
