\section{Diskussion}
\subsection{Photolumineszenz-Eigenschaften der kolloidalen Nanokristalle}
Die in Abbildung (\ref{abb:auf1a}) zu sehenden Messwerte und Fit-Kurven stimmen gut \"{u}berein.
Daher k\"{o}nnen auch gute Messergebnisse aus diesen PL-Spektren erwartet werden.
\begin{table}
	\centering
	\caption{Experimentell ermittelte Wellenl\"{a}nge der Photoluminezsenz.}
\begin{tabular}{|r|c|}
	\hline
	{Probe} & {Wellenl\"{a}nge $\lambda$ / nm} \\
	\hline
	1	& 642,8(19)  \\
	2	& 580,8(32)  \\
	3	& 540,2(87)  \\
	\hline
\end{tabular}
\label{tab:disku1}
\end{table}
In Tabelle (\ref{tab:disku1}) sind die experimentell bestimmten Wellenl\"{a}ngen aufgelistet.
Die berechneten Radien der Nanopartikel der einzelnen Proben liegen wie erwartet im $nm$-Bereich (siehe Tabelle (\ref{tab:disku2})).
\begin{table}
	\centering
	\caption{Experimentell ermittelte Gr\"{o}{\ss}e der Nanopartikel.}
\begin{tabular}{|r|c|}
	\hline
	{Probe} & {Radius $r_{NP}$ / nm} \\
	\hline
	1	&	7,717	\\
	2	&	5,338	\\
	3	&	4,502	\\
	\hline
\end{tabular}
\label{tab:disku2}
\end{table}

\bigskip
Die Messung in Abbildung (\ref{abb:polarisation}) hat gezeigt, dass keine Polarisation der Photolumines{\-}zenz vorliegt.
Eine m\"{o}gliche Erkl\"{a}rung k\"{o}nnten die Quantenpunkte liefern, welche in der fl\"{u}ssigen Probe alle eine zuf\"{a}llige Orientierung aufweisen.
Auch w\"{a}re es denkbar, dass die kugelf\"{o}rmige Struktur der Nanokristalle eine Rolle spielt.
Denn durch die kugelf\"{o}rmige Struktur besitzen sie keine Achse zu der sie polarisiert werden k\"{o}nnen.
%Polarisation zerf\"{a}llt durch Relaxationseffekte 

\bigskip 
Bei den leistungsabh\"{a}ngigen PL-Spektren ist f\"{u}r die Intensit\"{a}t der Emissionspeaks ein lineares Verhalten festzustellen (siehe Abbildung (\ref{abb:Leistungen_fit})).
Allerdings muss hier ber\"{u}cksichtigt werden, dass die Intensit\"{a}ten auf $1 \, $s normiert wurden, da bei unterschiedlichen Zeit{\-}inter{\-}vallen gemessen wurde, um eine S\"{a}ttigung des Detektors zu vermeiden.
Das Skalieren dieser Zeitintervalle sollte idealerweise linear erfolgen, um die Messergebnisse nicht zu verf\"{a}lschen.
Die Intensit\"{a}t der Emissionspeaks sollte mit steigender Laserleistung nicht mehr linear von dieser abh\"{a}ngen, da die M\"{o}glichkeit besteht, dass mehrere Exzitonen gleichzeitig in einem Quantenpunkt enstehen.
Auch kann es vorkommen, das mehrere Photonen absorbiert werden um ein einziges Exiton zu erzeugen.

Die Abh\"{a}ngigkeit der Emissions{\-}wellen{\-}l\"{a}nge von der angelegten Laserleistung ist anhand der Ergebnisse in Abbildung (\ref{abb:auf1c_ergebnisse}) nicht eindeutig zu bestimmen.
Durch die Exsistenz mehrerer Exzitonen in einem Quantenpunkt k\"{o}nnen sich zus\"{a}tzlich auch Biexzitonen ausbilden.
Diese k\"{o}nnen zu einer Veränderungen der Emissionswellenlänge beitragen.
Auch l\"{a}sst sich die ver\"{a}nderliche Emissionswellenlänge durch im Material stattfindende, intensitätsabhängige, nichtlinearen Effekte erklären.

Auch kann nicht gesagt werden, ob die Lienenbreite in eine S\"{a}ttigung geht und sich um den Wert $19,7$ einpendelt oder doch mit steigender Laserleistung noch weiter sinkt.
%Das konvergierende Verhalten der Halbwertsbreiten ist ebenfalls durch die Sättigung der Probe zu erklären. 
Hierf\"{u}r m\"{u}ssten mehr Messungen mit einem kleinen Abstand der Laserleistung durchgef\"{u}hrt werden, um ein aussagekr\"{a}ftiges Ergebnis zu erzielen.
%Die grundsätzliche spektrale Verbreiterung der PL gegenüber dem anregenden Licht ist auf leicht unterschiedliche Quantenpunktgrößen innerhalb einer Probe zurückzuführen.

\subsection{Abhängigkeit der Photolumineszenz von der Laserwellenlänge}
Die aufgenommenen Messungen in Abbildung (\ref{abb:auf2P1}) bis (\ref{abb:auf2P3}) zeigen f\"{u}r die drei Proben sehr unterschiedliche Ergebnisse.
So ist beispielsweise bei einer Anregungswellenl\"{a}nge von $636 \, \text{nm}$ keine Photolumineszenz bei den Proben 2 und 3 festzustellen.
Vermutlich muss, um Photolumineszenz zu erzeugen, eine kleinere Anregungswellenl\"{a}nge benutzt werden, als die Photolumineszenzwellenl\"{a}nge der verwendeten Probe.
Allgemein ist ein kein eindeutiges Verhalten der Photolumineszenz mit variierendem Verhaeltnis von Anregungs- und Photolumineszenzwellenl\"{a}nge zu erkennen.
%Ansonsten lassen sich keine Zusammenh\"{a}nge zwischen Anregungs- und Photolumineszenzwellenl\"{a}nge erschlie{\ss}en.

\subsection{Linearer Polarisationsgrad von Flüssigkeiten (Bonus-Aufgabe)}
Da bei der Messreihe im letzen Abschnitt der Auswertung die Reflexionspeaks eine deutlich h\"{o}here Intensit\"{a}t aufweisen, als die Emissionspeaks, fallen letztere relativ klein aus.
Vermutlich w\"{a}re der Polarisationsgrad bei l\"{a}ngeren Messzeiten deutlicher zu erkennen, es soll jedoch eine S\"{a}ttigung des Detektors vermieden werden.
Trotz der nicht sehr stark ausgepr\"{a}gten Photolumineszenzen konnte der Polarisationsgrad f\"{u}r $0^{\circ}$ und $90^{\circ}$ Anregungspolarisation bestimmt werden.
\begin{align*}
	\text{P}_{0^{\circ}} &= 0,08204 \\
	\text{P}_{90^{\circ}} &= 0,03779 \\	
\end{align*}
Beide Werte liegen in der selben Gr\"{o}{\ss}enordnung.
${P}_{0^{\circ}}$ ist dennoch gr\"{o}{\ss}er als ${P}_{90^{\circ}}$.
Im Vergleich zu den ersten drei untersuchten Nanokristall-Proben liegen in der Wein-Probe keine kugelf\"{o}rmigen Nanokristalle, sondern verschiedenen Molek\"{u}lstrukturen vor.
Diese k\"{o}nnen an einer Achse polarisiert werden.
%Zudem handelt es sich um ein Fl\"{u}ssigkeitsgemisch

\bigskip
Zusammenfassend l\"{a}sst sich sagen, dass der Versuchsaufbau des Experimentes gut geeignet ist um die Photolumineszenz-Eigenschaften der kolloidalen Nanokristalle zu bestimmen.
So konnte die Gr\"{o}{\ss}en der Quantenpunkte mithilfe von Photolumineszenz{\-}mess{\-}un{\-}gen gut abgesch\"{a}tzt werden und auch die unterschiedliche Abhängigkeiten der Photo{\-}lumines{\-}zenz von der angelegten Laserleistung untersucht werden.

