\section{Diskussion}
\subsection{Photolumineszenz-Eigenschaften der kolloidalen Nanokristalle}
Die in Abbildung (\ref{abb:auf1a}) zu sehenen Messwerte und Fit-Kruven stimmen gut \"{u}ber ein.
Daher k\"{o}nnen auch gute Messergebnisse aus diesen PL-Spektren erwartet werden.
\begin{table}
	\centering
	\caption{Experimentell ermittelte Wellenl\"{a}nge der Photoluminezsenz.}
\begin{tabular}{|r|ccc|}
	\hline
	{Probe} & \multicolumn{2}{c}{Wellenl\"{a}nge $\lambda$ / nm} & {Abweichung / \%} \\
	 & Theorie & experimentell &  \\
	\hline
	1	&	644	& 642,819 &	0,1834 \\
	2	&	580	& 580,832 &	0,1434 \\
	3	&	542	& 540,287 &	0,3161 \\
	\hline
\end{tabular}
\label{tab:disku1}
\end{table}
Wie in Tabelle (\ref{tab:disku1}) aufgelistet, sind die Abweichungen von der experimentell bestimmten Wellenl\"{a}nge sehr gering.
Auch die berechneten Radien der Nanopartikel der einzelnen Proben liegen wie erwartet im $nm$-Bereich (siehe Tabelle (\ref{tab:disku2})).
\begin{table}
	\centering
	\caption{Experimentell ermittelte Gr\"{o}{\ss}e der Nanopartikel.}
\begin{tabular}{|r|c|}
	\hline
	{Probe} & {Radius $r_{NP}$ / nm} \\
	\hline
	1	&	7,717	\\
	2	&	5,338	\\
	3	&	4,502	\\
	\hline
\end{tabular}
\label{tab:disku2}
\end{table}

\bigskip
Die Messung in Abbildung (\ref{abb:polarisation}) hat gezeigt, dass keine Polarisation der Photolumineszenz vorliegt.
Eine m\"{o}gliche Erkl\"{a}rung k\"{o}nnten die Quantenpunkte liefern, welche in der fl\"{u}ssigen Probe alle eine zuf\"{a}llige Orientierung aufweisen.
Auch w\"{a}re es denkbar, dass die kugelf\"{o}rmige Struktur der Nanokristalle eine Rolle spielt.
Denn durch die kugelf\"{o}rmige Struktur besitzen sie keine Achse zu der sie polarisiert werden k\"{o}nnen.
%Polarisation zerf\"{a}llt durch Relaxationseffekte 

\bigskip 
Bei den leistungsabh\"{a}ngigen PL-Sperktren ist f\"{u} die Intensit\"{a}t der Emissionspeaks ein linearen Verhalten festzustellen (siehe Abbildung (\ref{abb:Leistungen_fit})).
Die Abh\"{a}ngigkeit der Emissions{\-}ener{\-}gie von der angelegten Laserleistung ist anhand der Ergibnisse in Abbildung (\ref{abb:auf1c_ergebnisse}) nicht eindeutig zu bestimmen.
Auch kann nicht gesagt werden, ob die Lienenbreite in eine S\"{a}ttigung geht und sich um den Wert $19,7$ einpendelt oder doch mit steigender Laserleistung noch weiter sinkt.
Hierf\"{u}r m\"{u}ssten mehr Messungen mit einem kleinen Abstand der Laserleistung durchgef\"{u}hrt werden, um ein aussagekr\"{a}ftiges Ergebnis zu erzielen.

\subsection{Abhängigkeit der Photolumineszenz von der Laserwellenlänge}
Die in Abschnitt (\ref{sec:2}) aufgenommenen Messungen zeigen f\"{u}r die drei Proben sehr unterschiedliche Ergebnisse.
So ist beispielsweise bei einer Anregungswellenl\"{a}nge von $636 \, \text{nm}$ keine Photolumineszenz bei den Proben 2 und 3 festzustellen.
Vermutlich muss, um Photolumineszenz zu erzeugen, eine kleinere Anregungswellenl\"{a}nge benutzt werden, als die Photolumineszenzwellenl\"{a}nge der verwendeten Probe.
Ansonsten lassen sich keine Zusammenh\"{a}nge zwischen Anregungs- und Photolumineszenzwellenl\"{a}nge erschlie{\ss}en.

\subsection{Linearer Polarisationsgrad von Flüssigkeiten (Bonus-Aufgabe)}
Da bei der Messreihe in Abschnitt (\ref{sec:3}) die Reflexionspeaks eine deutlich h\"{o}here Intensit\"{a}t aufweisen, als die Emissionspeaks, fallen letztere relativ klein aus.
Vermutlich w\"{a}re der Polarisationsgrad bei l\"{a}ngeren Messzeiten deutlicher zu erkennen, es soll jedoch eine S\"{a}ttigung des Detektor vermieden werden.
Trotz der nicht sehr stark ausgepr\"{a}gten Photolumineszenzen konnte der Polarisationsgrad f\"{u}r $0^{\circ}$ und $90^{\circ}$ bestimmt werden.
\begin{align*}
	\text{P}_{0^{\circ}} &= 0,08204 \\
	\text{P}_{90^{\circ}} &= 0,03779 \\	
\end{align*}
Beide Werte liegen in der selben Gr\"{o}{\ss}enordnung.
${P}_{0^{\circ}}$ ist dennoch gr\"{o}{\ss}er als ${P}_{90^{\circ}}$.

\bigskip
Zusammenfassend l\"{a}sst sich sagen, dass der Versuchsaufbau des Experimentes gut geeignet ist um die Photolumineszenz-Eigenschaften der kolloidalen Nanokristalle zu bestimmen.
So konnte die Gr\"{o}{\ss}en der Quantenpunkte mithilfe von Photolumineszenz{\-}mess{\-}un{\-}gen gut abgesch\"{a}tzt werden und auch die unterschiedliche Abhängigkeiten der Photo{\-}lumines{\-}zenz von der angelegten Laserleistung untersucht werden.

