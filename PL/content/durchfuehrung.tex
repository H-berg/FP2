\section{Durchführung}
\label{sec:Durchführung}

\subsection{Aubau}
\label{sec:Aufbau}

In Abbildung \ref{fig:aufbau} ist zunächst eine Aufnahme des Versuchsaufbau zu
sehen. Dabei befindet sich oben links eine Halbleiter-Laserdiode($\SI{405}{\nano\meter}$).
Das entsendete Licht fällt zunächst in einen optischen Isolator/Diode, welche
unerwünschte Rückreflexionen unterbindet und damit dem Laser vor Störungen schützt.
Bei dem optischen Isolator wird der Faraday-Effekt genutzt, indem ein Faraday-Rotator,
welche auf die Wellenlänge von $\SI{405}{\nano\meter}$ eingestellt ist,
zwischen zwei um $\SI{45}{\degree}$ zueinandere verdrehten Polarisationsfiltern
hintereinandere geschaltet werden. Licht aus dem Laser wird nun durch den Faraday-Rotator
um $\SI{45}{\degree}$ gedreht und passiert den dahinter geschalteten Polarisationsfilter.
Licht aus der anderen Richtung wird hingegen so gedreht, dass es den Filter nicht
mehr passieren kann.
Der dem optischen Isolator hintergeschaltende Gradientenfilter hat die Aufgabe,
die Laserleistung zu regulieren, damit darauffolgende Bauteile nicht übersteuert
werden. Dabei lässt sich die ausgehende Intensität über Ändernung des Polarwinkels
regulieren. Dem Gradientenfilter folgen eine $\lambda$/2-Platte und ein
Glan-Taylor-Prisma, welche beide auf dem Effekt der Doppelbrechung basieren. Bei
der Doppelbrechung wird das Licht in einen außerordentlichen und in einen ordentlichen
Strahlanteil zerlegt, wobei beide Anteile nun linear polarisiert sind und deren
Polarisation senkrecht zueinander steht. Der ordentliche Anteil folgt dem
Snelliusschen Brechungsgesetz, wohingegen der außerordentliche Anteil diesem
eben nicht folgt. Beide Strahlanteile haben damit unterschiedliche Brechzahlen.
Die $\lambda$/2-Platte nutzt den Unterschied in der Brechzahlen, um die Phase
zwischen den beiden Anteilen um $\lambda$/2 zu drehen.

\begin{figure}[hbtp]
	\centering
	\includegraphics[width=\textwidth]{Abb/aufbau.png}
	\caption{Versuchsaufbau \cite{flex} zur Untersuchung der Photolumineszenz. Die
    hier dargestellte Weißlicht-LED-Quelle wird in diesem Versuch jeweils durch einen
    448, 518 und einem $\SI{636}{\nano\meter}$ Laser ausgetauscht. Ebenso wird in diesem
    Versuch auf die Sammellinse im Strahlengang verzichtet.}
	\label{fig:aufbau}
\end{figure}
\noindent
Ist die Polarisation
nun richtig eingestellt, lässt sich der ordentliche Strahlanteil rausfiltern und
nur der außerordentliche Strahl kann das Primsa passieren. Im letzten Schritt
durchläuft das Licht den dichroischen Spiegel, welcher nur Licht aber einer bestimmten
Wellenlänge passieren lässt. Ausgenutzt wird hier, dass sinnvoll hintereinander
geschaltete Medien mit unterschiedlichen Brechzahlen, Wellenlängen unterhalb von
$\SI{435}{\nano\meter}$ destruktiv und Wellenlänge oberhalb konstruktiv interferieren.
Nun trifft das linear polarisiert Licht der Wellenlänge größer als $\SI{435}{\nano\meter}$
die Probe und es kommt zu den in der Theorie \ref{sec:Theorie} benannten Effekten,
welche mittels des Detektionpfades gemessen wird. Dazu wird die entsendete
Photolumineszenz zunächst mit einer Sammellinse mit einer Brennweite von
$\SI{60}{\milli\meter}$ gebündelt. Dann durchläuft das Licht wieder die Kombination
aus $\lambda$/2-Platte und dem Glan-Taylor-Prisma und wird schließlich durch einen
Glasfaserkoppler in das CCD Spektrometer geleitet.
Im weiteren Verlauf des Versuches, wird ein weiterer Anregungspfad aufgebaut. Dabei
wird anstelle der in Abbildung \ref{fig:aufbau} dargestellten Weißlicht-LED-Quelle
jeweils ein Laser der Wellenlänge 448, 518 und $\SI{636}{\nano\meter}$
aufgestellt. Der Silberspiegelt erlaubt es dann, das Licht auf den dichroischen Spiegel
und damit auf die Probe zu leiten.\\

\subsection{Experiment}
\label{sec:Experiment}
