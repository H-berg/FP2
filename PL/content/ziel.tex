\section{Einleitung und Zielsetzung}
\label{sec:Ziel}
In diesem Versuch werden Halbleiter-Nanokristall-Quantenpunkte, welche mit Methoden
aus der kolloidalen Chemie hergestellt werden, untersucht. Entdeckt wurden sie erstmals
im Jahre 1980 \cite{eki}. Wie der Name bereits verrät, handelt es sich hier um
Kristalle mit einem Durchmesser in der Größenordnung von $\SI{1}{\nano\meter}$ bis
$\SI{100}{\nano\meter}$ (etwa 5-500 Atome). Aufgrund ihrer geringen Größe treten
quatenmechanische Eigenschaften zum vorschein, welche in der Medizin als bildgebendes
Verfahren \cite{biom}, in der Informatiosübertragung für die Kryptografie \cite{cryp}
und als Qubits für Quantencomputer \cite{compt}, in der Displayindustrie als QLED
\cite{qled} und in weiteren Gebieten genutzt werden können.\\
Im Folgenden werden für drei verschiedene Proben die Größe der Nanokristalle aus
der Emissionsenergie abgeschätzt, die Polarisation der Photolumineszenz untersucht
und die Photolumineszenz-Spektren in Abhängigkeit der einstrahlenden Laserleistung
und -wellenlänge gemessen. Zuletzt wird das Photolumineszenz-Spektrum einer Wein-Probe
untersucht.
