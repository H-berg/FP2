\section{Theorie}
\label{sec:Theorie}

%\subsection{Nanohalbleiterkristall}
%\label{sec:Nanohalbleiterkristall}

Die kolloidale Synthese erlaubt es II-VI-Halbleiter Nanokristalle herzustellen.
Dazu werden unter einer Schutzgasatmosphäre (z.B. Argon) Cadmium und Selen in
eine um mehr als $\SI{200}{\degreeCelsius}$ aufgeheiztes Dispersionsmedimum
gelöst. Bei diesem Prozess bilden sich in dem Dispersionsmedium Nanokristalle
aus CdSe und können kontrolliert bis zu einer gewünschten Kristallgröße wachsen.
Im nächsten Schritt werden die Nanokristalle mit einem weiteren II-VI-Halbleiter
mit größerer Bandlücke ummantel, indem Zinksulfid (ZnS) in die Lösung gegeben wird.
Damit isoliert die ZnS-Schale den CdSe-Kern von der Umwelt (siehe Abbildung
\ref{fig:ZnS_CdSe}). Da der Nanokristall nahezu kugelförmig ist, handelt es sich
hier um einen null dimensionalen Nanokristall. Zuletzt werden die Nanokristalle
aus dem Dispersionsmedimum ausgefällt \cite{komp}.\\

\begin{figure}[hbtp]
	\centering
	\includegraphics[width=0.5\textwidth]{Abb/ZnS_CdSe.png}
	\caption{Schematische Darstellung eines kolloidalen Typ-I CdSe/ZnS Quantenpunkts
   und seiner Energiestruktur \cite{anleitung}.}
	\label{fig:ZnS_CdSe}
\end{figure}
\noindent
Wird einem Nanokristall Energie in Form von Laserlicht zugeführt, so kann ein
Elektron aus dem Valenzband in das Leitungsband angehoben werden. Dabei hinterlässt
das Elektron ein Defekt, welches eine positive Ladung zugeschrieben werden kann.
Sowohl das Elektron als auch das Defektelektron relaxieren in einen energetisch
günstigeren Zustand nah der Bandlücke. Die überschüssige Energie wird dabei in
strahlungsloser Form an Phononen, Defekten, Störstellen oder anderen Ladungsträger
abgegeben. Da die Bandlücke zu groß ist, entsteht durch die Bindung von Elektron
und Defektelektron ein Exziton. Nach hinreichender Zeit kann es zu Rekombination
von Elektron und dem Defektelektron kommen, wodurch ein Photon mit der
Rekombinationsenergie abzüglich der Exzitonenbidungsenergie emittiert wird.
Es kommt zur Photolumineszenz. Ist die Energie des Laserlichtes größer als die
des CdSe-Kerns ($\SI{2,0}{\electronvolt}$) und kleiner als die der ZnS-Schale
($\SI{3,6}{\electronvolt}$), so entsteht in der ZnS-Schale zunächst ein
Elektron-Loch-Paar. Da es sich hier um einen Quantenpunkt Typ-I handelt, kann das
Elektron und das Defektelektron an die Bandlücke des CdSe-Kerns relaxieren.
In Abbildung \ref{fig:photoqp}(a) ist illustriert, wie der Quantenpunkt das
so entstandene Exziton einfängt. Befindet sich ein Elektron bereits im ZnS-Leitungsband,
wie in Abbildung \ref{fig:photoqp}(b) dargestellt, so ist von einem geladenen
Exziton die Rede. Ist die Laserleistung hoch, können sich auch Exzitonen auf höheren
Energieniveaus befinden, da eine Relaxation in ein tieferes Niveau nicht mehr
mögich ist (siehe dazu Abbildung \ref{fig:photoqp}(c)). Allerdings kann es bei
zu hohen Laserleistungen, und damit zu vermehrte Coulomb-Anziehung zwischen
den Elektron-Loch-Paaren, zur Veränderung der Bandstruktur kommen.

\begin{figure}[hbtp]
	\centering
	\includegraphics[width=0.8\textwidth]{Abb/photoqp.png}
	\caption{Schematische Darstellung des Einfangprozesses eines Exziton in einen
	Quantenpunkt (a), eines geladenen Exzitons (b) und einer Relaxation
	aus höheren Energieniveaus bedingt durch vermehrter Anzahl von Exzitonen (c) \cite{lars}.}
	\label{fig:photoqp}
\end{figure}
\noindent
Wie oben erwähnt, kann das Wachstum der Nanokristalle kontrolliert werden und
somit unterschiedlich große Bandlücken erzeugt werden. Dementsprechend lässt
sich auch die Energie des beim Rekombinieren entstehenden Photons nach der Brus-Formel

\begin{equation}
	E_{\gamma} = \frac{h^2}{8R^2}\cdot \biggl(\frac{1}{m_{\text{e}}} + \frac{1}{m_{\text{d}}} \biggr)
								- \frac{1,8e^2}{4\pi\epsilon\epsilon_0 R},
	\label{F1}
\end{equation}
\noindent
und damit auch nach $E$ = $\frac{hc}{\lambda}$ die Wellenlänge $\lambda$ der
Photolumineszenz wählen. Hierbei definiert $R$ die Partikelgröße und
$m_{\text{e,d}}$ die effektive Masse des Elektrons bzw. des Defektelektrons.
Dieses Verhalten ist in Abbildung \ref{fig:size} schematisch dargestellt.

\begin{figure}[H]
\centering
	\begin{subfigure}[t]{0.4\textwidth}
	\includegraphics[width=\textwidth]{Abb/size.png}
	\end{subfigure}
	~
	\begin{subfigure}[t]{0.4\textwidth}
	\includegraphics[width=\textwidth]{Abb/size2.png}
	\end{subfigure}
	\caption{Schematische Darstellung des Verhalten zwischen Nanokristallgröße und
	Wellenlänge $\lambda$ der Photolumineszenz\cite{size}\cite{blink}.}
\label{fig:size}
\end{figure}
\noindent
Beim aussende von Licht kann es zu Intensitätsschwankungen kommen, welche als
Blinken wahrgenommen werden. Eine sehr unwahrscheinliche Ursache dafür kann die
Auger Autoionisation sein, bei der sich zwei Exzitonen (ein sogenannts Biexziton)
in einem Quantenpunkt befinden und eins davon rekombiniert. Die freigesetze Energie
kann das übrige Elektron aus dem Quantenpunkt heben (Abbildung \ref{fig:blink}($a_1$)).
Auch können thermische Ionisierungen (Abbildung \ref{fig:blink}($a_2$)) und
direktes Tunnel (Abbildung \ref{fig:blink}($a_3$)) Ursache für die
Intensitätsschwankungen sein. Ein Defektelektron bleibt zurück und damit löscht
die statistisch bevorzugte nichtstrahlende Auger-Rekombination (Abbildung
\ref{fig:blink}($b$)) die Photolumineszenz aus, indem die freigesete Energie in
Form von Wärme abgegeben wird.

bei dem Ladungsträger durch thermische Energie aus dem Nanopartikel gehoben werden.
\begin{figure}[hbtp]
	\centering
	\includegraphics[width=0.8\textwidth]{Abb/blink.png}
	\caption{Schematischer Darstellung der Intensitätsschwankungen. Dabei sind die
	in (a) beschriebenen ionisierenden Vorgänge nichtstrahlend. In (b) wird ein
	nichtstrahlende Rekombination und in (c) eine Neutralisation in den Ausgangszustand
	illustiert. Der bereits beschriebene Photolumineszenz-Vorgang ist (e) zu finden
	\cite{blink}.}
	\label{fig:blink}
\end{figure}
\noindent
